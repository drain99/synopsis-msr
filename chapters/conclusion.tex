\section{Conclusion}
\label{sec:conclusion}
As introduced in \cref{sec:syn-intro}, most of the current solutions
to the problem of equivalence checking between a functional specification
and a C program relies heavily on manually provided correlation, inductive
invariants as well as proof assistants for discharging said obligations.
While the size of programs considered in our work is quite small,
we hope the ideas in \toolName{} will help
automate the proofs for such systems to some degree.

Prior work on push-button verification of specific
systems \cite{fscq,hyperkernel,serval,verifiedBPF}
involves a combination of careful system design and
automatic verification tools like SMT solvers.
Constrained Horn Clause (CHC) Solvers \cite{CHCeq}
encode verification conditions of programs containing loops and recursion,
and raise the level of abstraction for automatic proofs.
Comparatively, \toolName{} further raises the level
of abstraction for automatic verification from
SMT queries and CHC queries to automatic discharge of
proof obligations involving \recursiveRelations{}.

A key idea in \toolName{} is the conversion of proof
obligations involving \recursiveRelations{} to
bisimulation checks. Thus, \toolName{} performs {\em nested}
bisimulation checks as part of a `higher-level'
bisimulation search. This approach of
identifying \recursiveRelations{} as invariants and using
bisimulation to discharge the associated
proof obligations may have applications
beyond equivalence checking.
