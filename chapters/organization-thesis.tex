\vspace{-12px}
\section{Outline of the Thesis}
\vspace{-10px}
\label{sec:outlinethesis}
\textbf{Chapter 1} of the thesis contains a general introduction to the research problem of verification C programs against a functional specification.
We take a C program and its analogue in a safe functional language, and contrast their differences.
We summarize our approach and finish with the major contributions.

\textbf{Chapter 2} begins with an introduction to a minimal function language `Spec' and an intermediate representation (IR).
The rest of this chapter provides a background on bisimulation relation and product program, as well as
introduce terminology used in the rest of the thesis.
We finish with a formal definition of equivalence.

\textbf{Chapter 3} starts with proof obligations and their properties.
The rest of the chapter gradually introduces our first contribution: A Proof Discharge Algorithm and related sub-procedures with the help
of two example programs introduced in the last two chapters. We also introduce a program representation of values, called `deconstruction program'.

\textbf{Chapter 4} contains a discussion on the two major components of our algorithm: (a) a counterexample-guided correlation algorithm
to search for a bisimulation relation and (b) a counterexample-guided invariant inference algorithm.
These two components along with our proof discharge algorithm allow automatic end-to-end equivalence checking.
We formalize handling of procedure calls, and finish with a dataflow formulation of a pointer analysis
used by our equivalence checker.

\textbf{Chapter 5} introduces a program graph representation of values, called `value graphs', similar to `deconstruction program'.
We motivate it by listing its advantages and give an algorithm to convert expressions to this representation.
This helps us simplify our proof discharge algorithm.

In \textbf{Chapter 6}, we introduce our automatic equivalence checker tool named \toolName{}, based on our proof discharge algorithm
and counterexample-guided search procedures.
\toolName{} is evaluated on a large variety of C programs involving lists, strings, trees and matrices.
This includes C programs taken from C library implementations as well as manually written programs. We show that our equivalence checker is able
to prove equivalence of a single specification with multiple C implementations, each varying in its data layout and algorithmic strategy.

Finally, \textbf{Chapter 7} discusses the limitations of our algorithm and draws comparison with some related work.
We note our key ideas and finish with potential improvements to our algorithm.