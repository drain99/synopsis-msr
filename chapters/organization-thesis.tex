\section*{Outline of the Thesis}
\label{sec:outlinethesis}
\textbf{Chapter 1} of the thesis contains a general introduction to the research problem of automatically verifying C functions against a functional specification.
We take a C program and its analogue in a safe functional language, and contrast their differences. This helps us motivate the need to solve this problem.
We finish by stating our contribution: a proof discharge algorithm.

In \textbf{Chapter 2}, we constrain the programs being considered by formulating the problem statement. This helps us clearly define the
subproblem being solved. Next we define the necessary context and terminology (e.g., equivalence) for the rest of the thesis.

\textbf{Chapter 3} starts with some basic concepts related to equivalence such as bisimulation and product program.
The rest of the chapter gradually introduces the proof discharge algorithm and related subprocedures while
going through two example program pairs for demonstration.

Next, we formalize previously discussed topics in \textbf{Chapter 4}. We begin with the description of our custom language `Spec'. This is followed by
algorithms required in tandem with our proof discharge algorithm for an automatic equivalence checker such as a best-first search algorithm
for finding a bisimulation relation and an automatic invariant inference procedure. We finish this chapter with a dataflow analysis formulation
of our pointer analysis.

In \textbf{Chapter 5}, we evaluate our automatic equivalence checker based on the proof discharge algorithm on a large variety of C programs involving
lists, strings, trees and matrices.
This includes C programs taken from C library implementations as well as manually written programs. We show that our equivalence checker is able
to prove equivalence of a single specification with multiple of C implementations, each varying in its data layouts and algorithmic
strategies.

Finally, \textbf{Chapter 6} concludes the research and discusses some related works. We note our major ideas and contributions finishing with
some potential future improvements to our algorithm.