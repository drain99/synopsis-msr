\section{Outline of the Thesis}
\label{sec:outlinethesis}
% \textbf{Chapter 1} of the thesis contains a general introduction and description of the research problem, in which we first explain the motivation for the research by outlining the global burden caused by brain stroke especially on the developing nations in order to establish our research objectives. 

% In \textbf{Chapter 2}, we review the available methods for clinically differentiating ischemic and hemorrhagic strokes in order to understand the advantages and disadvantages of previous techniques. We conduct a re-analysis of all included studies to correct their methodology and get an accurate estimate of their performance via quantitative comparison. 

% In the previous chapter, we identify the issues with the clinical scores, which attempt to resolve in \textbf{Chapter 3} by developing a machine learning based framework for stroke classification solely using clinical attributes. We incorporate MICE imputation within our framework to address the missing values. Additionally, we demonstrate how easy it is to obtain extremely high but misleading performance as a result of target leakage in the data, which the majority of researches do not perform. We further conduct analysis to determine the most critical clinical markers for stroke classification.

% To gain a better understanding of the behaviour of attributes in medical datasets and their relationships, we develop a more generalised oblique sparse factor model in \textbf{Chapter 4}, a tool commonly used for exploratory data analysis. We propose a doubly-penalized lasso-based objective function and use a gradient descent-based alternate minimization algorithm to optimise it for the retrieval of model parameters. Additionally, we conduct a simulation study to compare our model's performance to that of two existing oblique sparse factor models, \textit{fanc} and \textit{sparse principal component analysis} and further demonstrate our model's performance over stroke dataset used in Chapter 3. 

% We further propose that analysing stroke patients' gait and microexpressions using computer vision and deep learning models may aid in retrieving additional information for identifying and classifying stroke in resource-constrained settings. However, there is no protocol for visual datasets that can be analysed for automated stroke identification and classification. In \textbf{Chapter 5}, we develop a protocol for collecting data from ten tests in order to enable deeplearning and computer vision models. 

% \textbf{Chapter 6} concludes the research and summarises our major contributions. In the end, the scope of potential future research directions is discussed. 