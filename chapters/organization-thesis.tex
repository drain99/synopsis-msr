\vspace{-12px}
\section{Outline of the Thesis}
\vspace{-10px}
\label{sec:outlinethesis}
\textbf{Chapter 1} of the thesis contains a general introduction to the research problem of verification C programs against a functional specification.
We take a C program and its analogue in a safe functional language, and contrast their differences. This helps us motivate the problem and its solution.
We finish with our contributions.

In \textbf{Chapter 2}, we constrain the programs being considered by formulating the problem statement. This helps us define the scope of our solution.
We introduce a custom minimal functional language called `Spec' and define the necessary terminology used in the rest of the thesis.

\textbf{Chapter 3} starts with background on program equivalence, bisimulation relation and product program.
The rest of the chapter gradually introduces our first contribution: A Proof Discharge Algorithm and related subprocedures with the help
of two example programs. We also introduce a program representation of values, called `reconstruction program'.

Next, we formalize previously discussed topics in \textbf{Chapter 4}. We begin with a detailed description of our custom language `Spec'. This is followed by
algorithms required in tandem with our proof discharge algorithm for an automatic equivalence checker such as a best-first search algorithm
for finding a bisimulation relation and an automatic invariant inference procedure. We finish this chapter with a dataflow analysis formulation
of a pointer analysis used by our equivalence checker.

\textbf{Chapter 5} introduces a program graph representation of values, called `deconstruction procedure', similar to `reconstruction procedure' as introduced in \textbf{Chapter 3}.
We motivate it by listing its advantages and give an algorithm to convert values to this representation. We finish by reformulating our
proof discharge algorithm using the said representation.

In \textbf{Chapter 6}, we introdice our automatic equivalence checker tool named \toolName{}, based on prior work and our proof discharge algorithm.
\toolName{} is evaluated on a large variety of C programs involving lists, strings, trees and matrices.
This includes C programs taken from C library implementations as well as manually written programs. We show that our equivalence checker is able
to prove equivalence of a single specification with multiple of C implementations, each varying in its data layouts and algorithmic
strategies.

Finally, \textbf{Chapter 7} discusses the limitations of our algorithm and compares it with some related work. We note our major ideas and finish with
some potential future improvements to our algorithm.