\usepackage[utf8]{inputenc}
\usepackage{physics}
\usepackage{amsmath}
\usepackage{tikz}
\usepackage{mathdots}
\usepackage{yhmath}
\usepackage{cancel}
\usepackage{color}
\usepackage{siunitx}
\usepackage{array}
\usepackage{multirow}
\usepackage{amssymb}
\usepackage{gensymb}
\usepackage{tabularx}
\usepackage{extarrows}
\usepackage{booktabs}
\usepackage{fontspec}

\usetikzlibrary{fadings}
\usetikzlibrary{patterns}
\usetikzlibrary{shadows.blur}
\usetikzlibrary{shapes}
\usetikzlibrary{arrows.meta}
\usetikzlibrary{positioning}
\usetikzlibrary{calc,decorations.pathreplacing,shapes.misc}
\usetikzlibrary{decorations.pathmorphing}

\usepackage{stackengine}
\usepackage{backnaur}
\usepackage{xcolor}
\usepackage{mathtools}
\usepackage{accents}

\newcommand{\xmark}{\ding{55}}%
\newcommand{\rulesep}{\unskip\ \vrule\ }

\setlength{\textfloatsep}{6.0pt plus 1.0pt minus 1.0pt}
\setlength{\intextsep}{3.0pt plus 0.5pt minus 0.5pt}

\makeatletter
\newcommand{\oset}[3][0ex]{%
  \mathrel{\mathop{#3}\limits^{
    \vbox to#1{\kern-2\ex@
    \hbox{$\scriptstyle#2$}\vss}}}}
\makeatother

\newcommand{\toolName}{S2C}%
\newcommand{\ite}[3]{\ensuremath{#1?#2:#3}}%
\newcommand{\sumDtor}{\underline{\tt if}-\underline{\tt then}-\underline{\tt else}}%
\newcommand{\sumIs}[2]{\ensuremath{#1\ \mathrm{is}\ \cons{#2}}}%
\newcommand{\fieldPath}[2][\ensuremath{.}]{%
  \def\nextitem{\def\nextitem{#1}}% Separator
  \renewcommand*{\do}[1]{\nextitem{\field{##1}}}% How to process each item
  \docsvlist{#2}% Process list
}
\newcommand{\prodAccess}[2]{\ensuremath{#1.{\fieldPath{#2}}}}%
\newcommand{\sumIf}[1]{\underline{\tt if}\ \ensuremath{#1}}%
\newcommand{\sumThen}[1]{\underline{\tt then}\ \ensuremath{#1}}%
\newcommand{\sumElif}[1]{\underline{\tt elif}\ \ensuremath{#1}}%
\newcommand{\sumElse}[1]{\underline{\tt else}\ \ensuremath{#1}}%
\newcommand{\recursiveRelations}{recursive relations}%
\newcommand{\recursiveRelation}{recursive relation}%
\newcommand{\SpecL}{Spec}%
\newcommand{\indEq}{\ensuremath{\sim}}%
\newcommand{\indEqDepth}[1]{\ensuremath{\sim_{#1}}}%
\newcommand{\indEqUapprox}[1]{\ensuremath{\approx_{#1}}}%
\newcommand{\depthBound}[2]{\ensuremath{\Gamma_{#1}(#2)}}%
% \newcommand{\structPointer}[4]{\ensuremath{{#1} \oset{#2}{\rightarrow}_{\type{#3}} {\field{#4}}}}%
\newcommand{\structPointer}[4]{\ensuremath{{#1} \overset{#2}{\rightarrow}_{\type{#3}} {\field{#4}}}}%
\newcommand{\arrIndex}[4]{\ensuremath{#1[#2]_{#3}^{\type{#4}}}}%
\newcommand{\memRead}[3]{\ensuremath{#1[#2]_{\type{#3}}}}%
\newcommand{\memWrite}[4]{\ensuremath{#1[#2 \leftarrow #3]_{\type{#4}}}}%
\newcommand{\pointsTo}{\ensuremath{\rightsquigarrow}}%
\newcommand{\hoareTriple}[3]{\ensuremath{\{#1\}(#2)\{#3\}}}%
\newcommand{\lift}[3]{\ensuremath{{{\tt C#1}_{#2}^{\tt #3}}}}%
\newcommand{\lifted}[4]{\lift{#1}{#2}{#3}{\ensuremath{(#4)}}}%
\newcommand{\mapping}[2]{\ensuremath{#1 \! \mapsto \! #2}}%
\newcommand{\mem}{\ensuremath{\textnormal{\fontspec{STIX Two Math}\symbol{"1D55E}}}}%
\newcommand{\nonTerm}[1]{\ensuremath{\langle}#1\ensuremath{\rangle}}%
\newcommand{\term}[1]{{\tt #1}}%
\newcommand{\sv}[1]{\ensuremath{{\tt #1}_{S}}}%
\newcommand{\cv}[1]{\ensuremath{{\tt #1}_{C}}}%
\newcommand{\apc}[1]{\ensuremath{{\tt A#1}}}%
\newcommand{\bpc}[1]{\ensuremath{{\tt B#1}}}%
\newcommand{\spc}[1]{\ensuremath{{\tt S#1}}}%
\newcommand{\cpc}[1]{\ensuremath{{\tt C#1}}}%
\newcommand{\dpc}[1]{\ensuremath{{\tt D#1}}}%
\newcommand{\scpc}[2]{\ensuremath{{\tt S#1\!:\!C#2}}}%
\newcommand{\scpcinv}[2]{\ensuremath{{\phi_{\tt S#1:C#2}}}}%
\newcommand{\ddpc}[2]{\ensuremath{{\tt D#1\!:\!D#2}}}%
\newcommand{\comv}[1]{\ensuremath{{\tt #1}}}%
\newcommand{\fstv}[1]{\ensuremath{{\tt #1}^{fst}}}%
\newcommand{\sndv}[1]{\ensuremath{{\tt #1}^{snd}}}%
\newcommand{\sdef}{\ensuremath{{(\sprog{}\ {\tt def})}}}%
\newcommand{\cfits}{\ensuremath{{(\cprog{}\ {\tt fits})}}}%
\newcommand{\spath}[2][\ensuremath{{\rightarrow}}]{%
  \def\nextitem{\def\nextitem{#1}}% Separator
  \renewcommand*{\do}[1]{\nextitem{\spc{##1}}}% How to process each item
  \docsvlist{#2}% Process list
}
\newcommand{\cpath}[2][\ensuremath{{\rightarrow}}]{%
  \def\nextitem{\def\nextitem{#1}}% Separator
  \renewcommand*{\do}[1]{\nextitem{\cpc{##1}}}% How to process each item
  \docsvlist{#2}% Process list
}%
\newcommand{\dpath}[2][\ensuremath{{\rightarrow}}]{%
  \def\nextitem{\def\nextitem{#1}}% Separator
  \renewcommand*{\do}[1]{\nextitem{\dpc{##1}}}% How to process each item
  \docsvlist{#2}% Process list
}%
\newcommand{\pathpar}{\ensuremath{+}}%
\newcommand{\pathset}[2][\ensuremath{{\rightarrow}}]{%
  \def\nextitem{\def\nextitem{#1}}% Separator
  \renewcommand*{\do}[1]{\nextitem{{\tt ##1}}}% How to process each item
  \docsvlist{#2}% Process list
}
\newcommand{\scedge}[4]{\ensuremath{{\small (\scpc{#1}{#2}) \! \rightarrow \! (\scpc{#3}{#4})}}}%
\newcommand{\ddedge}[4]{\ensuremath{{\small (\ddpc{#1}{#2}) \! \rightarrow \! (\ddpc{#3}{#4})}}}%
\newcommand{\type}[1]{{\tt #1}}%
\newcommand{\ctype}[1]{{{\,\tt :\!#1}}}
\newcommand{\cons}[1]{{\tt #1}}%
\newcommand{\field}[1]{{{\tt #1}}}%
\newcommand{\mlr}[1]{{\ensuremath{\tt #1}}}%
\newcommand{\mlrf}[1]{{\ensuremath{\tt #1_1}}}%
\newcommand{\mlrs}[1]{{\ensuremath{\tt #1_{2+}}}}%
\newcommand{\lhs}{{\tt LHS}}%
\newcommand{\rhs}{{\tt RHS}}%
\newcommand{\typeof}[1]{{\tt typeof(#1)}}%
\newcommand{\sizeof}[1]{{\tt sizeof(#1)}}%
\newcommand{\offsetof}[2]{{\tt offsetof(#1,#2)}}%
\newcommand{\addrof}[1]{{\tt addrof(#1)}}%
\newcommand{\heapr}{{\ensuremath{\mathcal{H}}}}%
\newcommand{\sprog}{{\ensuremath{\mathcal{S}}}}%
\newcommand{\cprog}{{\ensuremath{\mathcal{C}}}}%
\newcommand{\dprog}{{\ensuremath{\mathcal{D}}}}%
\newcommand{\fdprog}{{\ensuremath{\mathcal{D}^{fst}}}}%
\newcommand{\sdprog}{{\ensuremath{\mathcal{D}^{snd}}}}%
\newcommand{\pre}{{\ensuremath{Pre}}}%
\newcommand{\post}{{\ensuremath{Post}}}%
\newcommand{\corrtuple}[4]{{\ensuremath{\langle #1, #2, #3, #4 \rangle}}}%
\newcommand{\ttedge}[2]{{\ensuremath{[#1 \! \rightarrow \! {\tt #2}]}}}%
\newcommand{\vtedge}[3]{{\ensuremath{[#1 \! \overset{#3}{\rightarrow} \! {\tt #2}]}}}%
\newcommand{\sumn}{\ensuremath{\circled{+}}}%
\newcommand{\prodn}{\ensuremath{\circled{\times}}}%
\newcommand{\vtse}[1]{{\ensuremath{\omega_{#1}\!}}}%
\newcommand{\oland}{\ensuremath{\underbar{\land}}}%
\newcommand{\xfer}{{\tt tf}}%
\newcommand{\execpath}{{\tt ep}}%
\newcommand{\ftrace}[1]{{\ensuremath{{\tt Ftrace}(#1)}}}%
\newcommand{\comp}[2]{{\ensuremath{{\tt Comp}_{#2}(e)}}}%
\newcommand{\retval}[1]{{\ensuremath{{\tt Retval}(#1)}}}%
\newcommand{\pathcond}[1]{{\ensuremath{{\tt Pathcond}(#1)}}}%
\newcommand{\execpaths}[3]{{\ensuremath{{\tt EP}_{#1}(#2,#3)}}}%

\newcommand{\keyword}[1]{{\ensuremath{\ \textnormal{\textbf{#1}}\ }}}%
\newcommand{\subst}[3]{{\ensuremath{\{ #2 \mapsto #3 \} #1}}}%
\newcommand{\strcat}{\ensuremath{\mathbin\Vert}}%
\newcommand{\strsep}{\ensuremath{\raisebox{0.5pt}{\underline{\phantom{x}}}}}%

\newcommand{\Tstrut}{\rule{0pt}{2.7ex}}         % = `top' strut
\newcommand{\Bstrut}{\rule[-1.2ex]{0pt}{0pt}}   % = `bottom' strut

\newcommand*{\circled}[1]{\tikz[baseline=(char.base)]{
                          \node[shape=circle,draw,inner sep=1pt] (char) {#1};}}
\newcommand*{\curved}[1]{\tikz[baseline=(char.base)]{
                      \node[shape=rounded rectangle,draw,inner sep=3pt] (char) {\footnotesize #1};}}
\newcommand*{\inv}[1]{\tikz[baseline=(char.base)]{
                      \node[shape=rounded rectangle,draw,inner sep=2pt] (char) {\scriptsize {\tt #1}};}}
\newcommand*{\pred}[1]{\fbox{\scriptsize {\tt #1}}}
\newcommand*{\tfbox}[1]{\fbox{\ensuremath{#1}}}
\newcommand*{\compacttfbox}[1]{\setlength{\fboxsep}{1pt}\fbox{\ensuremath{#1}}}

\newcommand{\invgrammar}{\ensuremath{\textnormal{\fontspec{STIX Two Math}\symbol{"1D53E}}}}%
\newcommand{\memregions}{\ensuremath{\textnormal{\fontspec{STIX Two Math}\symbol{"211D}}}}%
\newcommand{\pseudoregs}{\ensuremath{\textnormal{\fontspec{STIX Two Math}\symbol{"1D54A}}}}%
\newcommand{\typegrammar}{\ensuremath{\textnormal{\fontspec{STIX Two Math}\symbol{"1D54B}}}}%

\newcommand{\astcons}[1]{{\textcolor{myastral}{\cons{#1}}}}%
\newcommand{\olifield}[1]{{\textcolor{myolive}{\field{#1}}}}%magenta


\definecolor{myastral}{RGB}{46,116,181}
\definecolor{myolive}{named}{olive}
\definecolor{mygreen}{rgb}{0,0.6,0.2}
\definecolor{mygray}{rgb}{0.5,0.5,0.5}
\definecolor{myred}{rgb}{0.8,0,0.2}

\newcommand{\Guide}{Sorav Bansal}
\newcommand{\Auth}{Indrajit Banerjee}
\newcommand{\Entry}{(2020CSY7569)}
\newcommand{\SynopsisTitle}{Counterexample-Guided Verification of Imperative Programs Against Implementation Agnostic Functional Specification}
\newcommand{\ThesisTitle}{Counterexample-Guided Verification of Imperative Programs Against Functional Specification}

\tikzset{
    ttinner/.style args={#1:#2}{draw,circle,inner sep=0.2mm,label={#1,inner sep=0.1mm:\scriptsize #2}},
    ttleaf/.style={font=\footnotesize},
    ttannot/.style={near start,align=center,font=\scriptsize},
    position/.style args={#1:#2 from #3}{
        at=(#3.#1), anchor=#1+180, shift=(#1:#2)
    },
    show control points/.style={
        decoration={show path construction, curveto code={
                \draw [blue, dashed]
                    (\tikzinputsegmentfirst) -- (\tikzinputsegmentsupporta)
                    node [at end, cross out, draw, solid, red, inner sep=2pt]{};
                \draw [blue, dashed]
                    (\tikzinputsegmentsupportb) -- (\tikzinputsegmentlast)
                    node [at start, cross out, draw, solid, red, inner sep=2pt]{};
            }
        },
        postaction=decorate
    },
}
