\begin{figure}
\begin{tabular}{cc}
\begin{subfigure}[b]{0.50\textwidth}
\includegraphics[scale=0.75]{figClistCfg.pdf}
\caption{\label{fig:deconsProg}Deconstruction Program}
\end{subfigure}%
&
\begin{subfigure}[b]{0.50\textwidth}
\includegraphics[scale=0.75]{figClistProductCfg.pdf}
\caption{\label{fig:deconsPCFG}Decons-PCFG}
\end{subfigure}%
%&
%\begin{subfigure}[b]{0.17\textwidth}
%\includegraphics[scale=0.8]{figMallocPointsToGraph.pdf}
%\caption{\label{xxx}XXX}
%\end{subfigure}%
\\
\end{tabular}
\vspace{-8px}
\caption{\label{fig:decons}The deconstruction program and the decons-PCFG for {\tt Clist$^{lnode}_m({\tt l_{C}})$}. In \cref{fig:deconsProg}, {\tt D0} represents the unrolling procedure entry node, and the square boxes show the transfer functions of the unrolling procedure (\cref{eqn:clist}). The dashed edges represent a recursive function call. In \cref{fig:deconsPCFG}, the square box to the right of node {\tt D0:D0} contains the inferred invariants for this decons-PCFG.}
\vspace{-8px}
\end{figure}
