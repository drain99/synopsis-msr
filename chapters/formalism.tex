\section{Formalism}
\label{sec:formalism}
\subsection{The \SpecL{} Language}
\label{sec:speclang}
We briefly discuss the properties of the \SpecL{} language in this section.
\SpecL{} supports first order algebraic data types (ADT) similar to functional languages.
The language also borrows its expression grammar heavily from functional languages.
This includes the usual constructs like {\tt let-in}, {\tt if-then-else}, function application and the {\tt match} statement
for pattern-matching (i.e. deconstructing) sum and product values.
\SpecL{} is equipped with a special {\tt assuming-do} construct for explicitly providing UB conditions as discussed in \cref{sec:eval}.
\SpecL{} also provides the typical bitvector and boolean operators such as {\tt bvadd(+)} for adding two bit-vectors, {\tt bvextract$_{a:b}$}
for extracting bits $a$ through $b$ and {\tt $\lessgtr_{s}$} for comparing bit-vectors interpreted as signed integers.


\begin{footnotesize}
\begin{bnf*}
\bnfprod{expr}{\bnfts{if} \bnfsp \bnfpn{expr} \bnfsp \bnfts{then} \bnfsp \bnfpn{expr} \bnfsp \bnfts{else} \bnfsp \bnfpn{expr}} \\
\bnfmore{\bnfts{let} \bnfsp \bnfpn{id} \bnfsp \bnfts{=} \bnfsp \bnfpn{expr} \bnfsp \bnfts{in} \bnfsp \bnfpn{expr}} \\
\bnfmore{\bnfts{match} \bnfsp \bnfpn{expr} \bnfsp \bnfts{with} \bnfsp \bnfpn{match-clause}^*} \\
\bnfmore{\bnfts{assuming} \bnfsp \bnfpn{expr} \bnfsp \bnfts{do} \bnfsp \bnfpn{expr}} \\
\bnfmore{\bnfpn{id} \bnfsp \bnfts{(} \bnfsp \bnfpn{expr}^* \bnfts{)}} \\
\bnfmore{\bnfpn{data-cons} \bnfsp \bnfts{(} \bnfsp \bnfpn{expr}^* \bnfts{)}} \\
\bnfmore{\bnfpn{expr} \bnfsp \bnfts{is} \bnfsp \bnfpn{data-cons}} \\
\bnfmore{\bnfpn{expr} \bnfsp \bnfpn{intrinsic-op} \bnfsp \bnfpn{expr}} \\
\bnfmore{\bnfpn{literal$_{\tt unit}$} \bnfor \bnfpn{literal$_{\tt bool}$} \bnfor \bnfpn{literal$_{\tt i<N>}$}} \\
\\
\bnfprod{match-clause}{\bnfts{|} \bnfsp \bnfpn{data-cons} \bnfsp \bnfpn{id}^* \bnfsp \bnfts{=>} \bnfsp \bnfpn{expr}} \\
\bnfprod{intrinsic-op}{\bnfpn{arithmatic-op} \bnfor \bnfpn{relational-op} \bnfor \bnfpn{logical-op}} \\
\\
\bnfprod{arithmatic-op}{\bnfts{+} \bnfor \bnfts{-} \bnfor \bnfts{*}  \bnfor \bnfts{bvextract$\mathrm{_{a:b}}$} \bnfor \bnfts{bvconcat}} \\
\bnfprod{relational-op}{\bnfts{=$_{\tt \{ unit,bool,i<N> \}}$} \bnfor \bnfts{<$_{\tt \{ s,u \}}$} \bnfor \bnfts{$\leq$$_{\tt \{ s,u \}}$} \bnfor \bnfts{$\geq$$_{\tt \{ s,u \}}$} \bnfor \bnfts{>$_{\tt \{ s,u \}}$}} \\
\bnfprod{logical-op}{\bnfts{not$_{\tt \{ bool,i<N> \}}$} \bnfor \bnfts{and$_{\tt \{ bool,i<N> \}}$} \bnfor \bnfts{or$_{\tt \{ bool,i<N> \}}$} \bnfor \bnfts{xor$_{\tt \{ bool,i<N> \}}$}} \\
\\
\bnfprod{literal$_{\tt unit}$}{\bnfts{()}} \\
\bnfprod{literal$_{\tt bool}$}{\bnfts{false} \bnfor \bnfts{true}} \\
\bnfprod{literal$_{\tt i<N>}$}{\bnfts{0} \bnfsk \bnfts{2$^{\tt N}$-1}} \\
\end{bnf*}
\end{footnotesize}

\subsection{Modeling Procedure Calls}
\label{sec:modelfcalls}
A top-level procedure $\delta$
in $S$ or $C$ may make non-tail recursive
calls, e.g., for traversing
a tree data structure.
Our correlation algorithm (\cref{sec:searchAlgoInformal}) ensures
that the anchor nodes around such a callsite are correlated
one-to-one across both programs. For example, let there be a
recursive call in $S$ at PC $A_S$, i.e., $A_S$ is the callsite.
Then we denote the program points just before and just
after this callsite
as $A^{{\tt b}}_S$ and $A^{{\tt a}}_S$ respectively.
Let {\tt args$_{A_S}$}
represent the values of the
actual arguments of this procedure call.
Let {\tt ret$_{A_S}$} represent the values returned by this procedure call.
Similarly, for a procedure call at PC $A_C$ in $C$, let
$A^{{\tt b}}_C$,
$A^{{\tt a}}_C$,
{\tt args$_{A_C}$} and {\tt ret$_{A_C}$} 
represent the before-callsite program point, after-callsite program point,
arguments and return values respectively.
Our algorithm ensures that the only correlations possible in
a product-CFG $\pi$ for these $S$ and $C$ program points are
{\tt $A_{\pi}^{{\tt b}}$=($A_{S}^{{\tt b}}$,$A_{C}^{{\tt b}}$)}
and
{\tt $A_{\pi}^{{\tt a}}$=($A_{S}^{{\tt a}}$,$A_{C}^{{\tt a}}$)}.

Recall that the recursive call at $A_S$ (or $A_C$) must
be a call to the top-level procedure $\delta$.
We utilize the user-supplied $Pre$ and $Post$ conditions for $\delta$ to obtain
the desired invariants at nodes $A_{\pi}^{{\tt b}}$
and 
$A_{\pi}^{{\tt a}}$ in the product-CFG.
We require a successful proof to
{\em ensure} that {\small $Pre(A^{{\tt args_S}}_S, A^{{\tt args_C}}_C, m_{\tt b})$}
holds at $A_{\pi}^{{\tt b}}$.  Further, the proof can
{\em assume} that {\small $Post(A^{{\tt ret_S}}_S, A^{{\tt ret_C}}_C, m_{\tt a})$} holds
at 
$A_{\pi}^{{\tt a}}$. Here, $m_{\tt b}$
and $m_{\tt a}$ represent the memory states in $C$ at
$A^{{\tt b}}_C$ and
$A^{{\tt a}}_C$ respectively.
Thus, for such recursive calls to the
top-level function, we inductively
prove the precondition (on the
arguments of the procedure call) at $A_{\pi}^{{\tt b}}$
and assume the postcondition (on the return values of the procedure call)
at $A_{\pi}^{{\tt a}}$.



\subsection{Invariant Inference and Counterexample Generation}
\label{sec:invinference}

\begin{table}[H]
\begin{center}
\caption{\label{tab:dataflow_formulation}Dataflow formulation for the Invariant Inference Algorithm.}
\setlength{\belowcaptionskip}{-30pt}
\begin{footnotesize}
\begin{tabular}{|l|l|}
\hline
Domain &
\makecell[c]{
\ldelim\{{2}{3mm}{\footnotesize \ \  $\phi_n$ is a conjunction of predicates drawn from\ \ }\rdelim\}{2}{3mm} \\ \ {\footnotesize grammar in \ref{fig:invGrammar}, $\Gamma_n$ is a set of counterexamples}} \\
\hline
Direction & Forward\\
\hline
\makecell[l]{Transfer function across \\ edge $e=(s\rightarrow d)$} & $(\phi_d,\Gamma_d) = f_e(\phi_s,\Gamma_s)$ (\cref{algo:tf})\\
\hline
\begin{tabular}{@{}l@{}}
Meet operator $\otimes$\\
{\footnotesize $(\phi_n,\Gamma_n) \leftarrow (\phi^1_n,\Gamma^1_n) \otimes (\phi^2_n,\Gamma^2_n)$}\\
\end{tabular}
& $\Gamma_n \leftarrow \Gamma^1_n\cup \Gamma^2_n$, \ \ \ \ \ \ \ \ $\phi_n \leftarrow \mathrm{\it StrongestInvCover}(\Gamma_n)$\\
\hline
Boundary condition & {\tt out[$n^{start}$]} = {\tt $(Pre,\Gamma_{n^{start}})$}\ \ \ \ \ \\
\hline
Initialization to $\top$ & {\tt in[$n$] = $($False,$\{\})$} for all non-start nodes\\
\hline
\end{tabular}
\end{footnotesize}
\end{center}
\end{table}
\vspace{-10px}

\Cref{tab:dataflow_formulation} presents our dataflow analysis for inferring invariants $\phi_n$
at each node $n$ of a product-CFG, while also generating a set of counterexamples $\Gamma_n$ at node $n$
that represents the potential concrete machine states at $n$.

Given the invariants and counterexamples at node $s$
($\phi_s,\Gamma_s$),
the transfer function initializes the new candidate set
of counterexamples at $d$ ($\Gamma^{can}_{d}$)
to the current set of counterexamples at $d$ ($\Gamma_{d}$) union-ed with
the counterexamples obtained by executing $\Gamma_s$ on edge $e$ ({\tt exec$_e$}).
The candidate invariant at $d$ ($\phi^{can}_d$) is computed
as the strongest cover of $\Gamma^{can}_{d}$ ({\em StrongestInvCover()}).
At each step, the transfer function attempts to prove
$\{\phi_s\} (e) \{\phi^{can}_d\}$ (by checking {\tt SAT}isfiability of its
negation). If the proof succeeds, the candidate invariant $\phi^{can}_d$ is
returned alongwith the counterexamples $\Gamma^{can}_d$ learned so far.
Else the candidate invariant $\phi^{can}_d$ is weakened using
the counterexamples obtained from the {\tt SAT} query ($\gamma$) and the proof attempt is
repeated.

The predicate grammar allows the automatic inference of
affine relations between bitvector values of both
programs, and \recursiveRelations{} between an ADT value in \SpecL{} ($\alpha_S$)
and a {\em lifted} ADT value from C (${\tt liftC}_m(p_C)$).
We enumerate these \recursiveRelation{}
guesses for all bitvector variables $v^{C}$ in $C$. For each variable $v^{C}$,
all candidate {\tt liftC} constructors are attempted. In
our implementation, the candidate
{\tt liftC}
constructors are derived from the constructors
present in the precondition $Pre$
and the postcondition $Post$.  More
sophisticated strategies for automatic guessing of
these lifting constructors
are possible.

{\em StrongestInvCover()} for affine relations involves
identifying the basis vectors of the kernel of the
matrix formed by the counterexamples in the bitvector
domain \cite{esop05,semalign}.
For inequality and
\recursiveRelations{}, {\em StrongestInvCover($\Gamma$)}
returns false iff any counterexample in $\Gamma$ evaluates the
relation to false --- this effectively simulates the Houdini approach \cite{houdini}.

\begin{figure}[t]
\begin{center}
\begin{subfigure}[t]{.46\textwidth}
\begin{algorithm}[H]
\begin{footnotesize}
\SetAlgoLined
\SetKwProg{Fn}{Function}{}{end}
\Fn{$f_e(\phi_s, \Gamma_s)$}{
  $\Gamma^{can}_{d} \mapsfrom \Gamma_{d} \cup {\tt exec}_e(\Gamma_s)$;\\
  $\phi^{can}_{d} \mapsfrom \mathrm{\it StrongestInvCover}(\Gamma^{can}_{d})$;\\
  \While{{\tt SAT$(\neg(\{\phi_s\} (e) \{\phi^{can}_{d}\}), \gamma_s)$}}
  {
    $\gamma_{d} \ \ \ \ \mapsfrom {\tt exec}_e(\gamma_s)$;\\
    $\Gamma^{can}_{d} \mapsfrom \Gamma^{can}_{d}\cup\gamma_{d}$;\\
    $\phi^{can}_{d} \mapsfrom \mathrm{\it StrongestInvCover}(\Gamma^{can}_{d})$;\\
  }
  \Return{$(\phi^{can}_{d}, \Gamma^{can}_{d})$;}\\
}
\end{footnotesize}
\end{algorithm}
\caption{\label{algo:tf} Transfer function $f_e$ across
edge $e=(s\rightarrow d)$.}
\end{subfigure}%
\hfill
\rulesep
\hfill
\begin{subfigure}[t]{.52\textwidth}
\begin{center}
\begin{tabular}{@{}l@{}l}
$Inv \rightarrow$ &
\begin{tabular}{@{}l|l|l@{}}
$\sum_{i}{c_iv_i}=c$ & $v_1 \odot v_2$ & $\alpha_S = {\tt liftC}_m(v^{C} \dots)$
\end{tabular}\\
\end{tabular}%
\vspace{7px}
\end{center}
\caption{\label{fig:invGrammar} Predicate grammar for constructing invariants.\\$v$ represents a bitvector variable in either $S$ or $C$. \\ $c$ represents a bitvector constant.\\$\odot$ represent one of $\{<,\leq\}$ inequality operators.\\$\alpha_S$ represents an ADT variable in \SpecL{}.\\$v^{C}$ represents a bitvector variable in $C$.\\$m$ represents the current $C$ memory state.}
\end{subfigure}%
\caption{Transfer function $f_e$ and Predicate grammar for invariant inference dataflow analysis in \cref{tab:dataflow_formulation}.
Given invariants ($\phi_{s}$) and counterexamples ($\Gamma_{s}$) at node $s$,
$f_e$ returns the updated
invariants ($\phi_{d}$) and counterexamples ($\Gamma_{d}$) at
node $d$.
{\em StrongestInvCover($\Gamma$)} computes the strongest invariant cover for counterexamples $\Gamma$.
{\tt exec$_e$($\Gamma$)} (concretely) executes
counterexamples $\Gamma$ over edge $e$.
{\tt SAT($\phi$, $\gamma$)} determines
the satisfiability of $\phi$; if satisfiable, the models (counterexamples) are returned in output parameter $\gamma$.}
\end{center}
\end{figure}


\subsection{Points-to Analysis}
\label{sec:pointsToFormal}
We first identify the set $R$ of all region labels.
For each call to {\tt malloc()} at PC $A$,
we add $A_1$ and $A_{2+}$ to $R$. $R=\bigcup_{A}\{A_1,A_{2+}\}\cup\{{\tt heap}\}$.

Let $S$ be the set of all scalar pseudo-registers in $C$'s
IR. We use a forward dataflow
analysis to identify a may-point-to function $\Delta: (S\cup{}R) \mapsto 2^{R}$
at each program point.
For an IR instruction {\tt x := c}, for constant $c$, the
transfer function updates $\Delta({\tt x}) := \emptyset$.
For instruction {\tt x := y op z} (for some arithmetic
or logical operand {\tt op}),
we update $\Delta({\tt x}) := \Delta({\tt y})\cup\Delta({\tt z})$.
For a
load instruction {\tt x := *y}, we
update $\Delta({\tt x})$ to $\bigcup_{R\in\Delta(y)}\Delta(R)$.
For a store instruction {\tt *x := y}, for all
$R\in\Delta({\tt x})$, we update $\Delta(R) := \Delta(R)\cup\Delta(y)$.
For recursive procedure calls, a {\em supergraph}
is created by adding control flow edges
from the call-site to the procedure head (copying actual arguments
to the formal arguments) and from the procedure
return to the returning point of the call-site (copying returned
value to the variable assigned at the callsite), e.g., in
\cref{fig:decons}, the dashed edges
represent supergraph edges.
For a malloc instruction {\tt x := malloc$_A$()} (where $A$
represents the allocation site), we perform the following steps (in order):
(a) Convert all existing occurrences of $A_1$ to $A_{2+}$, i.e., for all $r\in{}S\cup{}R$, if $A_1\in\Delta(r)$, then update $\Delta(r) := (\Delta(r)\setminus\{A_1\})\cup\{A_{2+}\}$;
(b) Update $\Delta({\tt x}) := \{A_1\}$;
(c)Update $\Delta(A_{2+}) := \Delta(A_{2+})\cup\Delta(A_1)$;
(d) Update $\Delta(A_1) := \emptyset$ (empty set).
% \begin{enumerate}
% \item Convert all existing occurrences of $A_1$ to $A_{2+}$, i.e., for all $r\in{}S\cup{}R$, if $A_1\in\Delta(r)$, then update $\Delta(r) := (\Delta(r)\setminus\{A_1\})\cup\{A_{2+}\}$.
% \item Update $\Delta({\tt x}) := \{A_1\}$
% \item Update $\Delta(A_{2+}) := \Delta(A_{2+})\cup\Delta(A_1)$.
% \item Update $\Delta(A_1) := \emptyset$ (empty set).
% \end{enumerate}

The meet operator is set-union. At program entry for
program $C$, $\Delta(r)=\{{\tt heap}\}$ for
all $r\in{}S\cup{}R$ (boundary condition at entry), where {\tt heap} represents all the {\em other} memory
regions that are not captured by the region labels associated with allocation
sites.

The same dataflow analysis is used for both the C program and a
reconstruction program for a lifting constructor.
The only difference is in the boundary condition at entry of a
reconstruction program. For the
reconstruction program, we use the results of the points-to
analysis on $C$ at the from-PC where the proof obligation is being
discharged; these results are used
to initialize the may-point-to function at the
entry of the reconstruction program.
% A deconstruction
% program does not contain any store and malloc operations.

%The allocation-site abstraction (with a bounded-depth call stack) is
%known to be effective at disambiguating memory regions belonging to
%different data structures
%\cite{allocationSiteAbstraction82,allocationSiteAbstraction90,allocationSiteAbstraction06}.
%In our work, we also need to reason about non-aliasing
%of the most-recently allocated object (through a {\tt malloc} call) and
%the previously-allocated objects (as in the {\tt List}
%construction example). The coarse-grained $\{1, 2+\}$
%categorization of allocation recency is effective for such
%disambiguation.