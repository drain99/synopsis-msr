\section{\large Problem Setting and Equivalence Definition}
\label{sec:problem-setting}
We restrict
our attention to programs that construct, read, and write to recursive
data structures in \SpecL{} and C.
If a \SpecL{} or C program contains multiple procedures, we first
convert all tail-recursive calls to loops, and then inline
all non-recursive procedure calls to obtain a single top-level procedure which is
compared for equivalence.
% The top-level procedures in $S$ and $C$
%must have the same name for our tool
%to compare them for equivalence.
A top-level procedure
may make recursive calls to itself (which are not tail recursive).
The C program
may also contain calls to memory allocation library functions
like {\tt malloc} whose
abstract semantics are available.

The inputs to
a \SpecL{} procedure
are its explicit program arguments, which may include
recursive data-structure values.  The inputs to a
C procedure include the explicit arguments passed to the C procedure (e.g., pointers)
and the implicit state of program memory at procedure entry.
Notice the difference in the nature of inputs to the two programs: while
\SpecL{} inputs are explicit well-typed values, C procedure's
inputs may be
derived from the state of the input memory (e.g., linked list formed by chasing the
{\tt next} pointer).
For checking equivalence, we require
the user to specify a precondition (at the entry of both
programs) that relates these two different
types of program inputs.
%The user may
%employ a
%\recursiveInvariant{} shape for specifying a precondition.

\Cref{fig:llAllocSpecIR} shows the Three-Address-Code (3AC) style
intermediate representation (IR) of the linked-list construction \SpecL{}
program in \cref{fig:llAllocSpec}.
We often omit intermediate registers in the intermediate representation
for brevity and ease of exposition, and refer to this as {\em abstracted} IR.
The primary differences between
the \SpecL{} source and IR are: (a) tail-recursive calls
are converted to loops in IR, and (b) {\tt match} statements
are converted to \sumDtor{} in IR,
where each branch of an \sumDtor{}
expression represents a distinct constructor.

Similarly, the C implementation is also lowered to a 3AC IR
that resembles LLVM IR \cite{llvm_langref_home}. The primary differences between a C
source and its IR are: (a) the sizes and
memory-layouts of both scalar (e.g., {\tt int}) and
compound (e.g., {\tt struct}) types are concretized in the IR,
and (b) we annotate any {\tt malloc} calls with
the IR PC at which that call appears
(e.g., {\tt malloc$_{\tt C4}$} in \cref{fig:llAllocCIR}).
%, and (c) the IR models only three types of UB
%present in the C source, namely out-of-bounds
%memory accesses (OOB), division-by-zero (DBZ), and alignment-related (Align)
%UB assumptions.
%We ignore the other types of
%UB for simplicity. If the user is interested in
%end-to-end verification of the executable code (compiled from
%the C implementation), then she must ensure that the
%the C compiler does not leverage these ignored UB types for
%optimizing the generated executable. This can be achieved, for example,
%by disabling compiler optimizations. Thus, we model only
%those essential types of UB (OOB, DBZ, Align) which may cause
%traps or errors during machine code execution, even in the absence
%of compiler optimization.

\toolName{} computes equivalence between the IR
of the \SpecL{} and C source programs.
Henceforth, we will omit the source
representation and only show the IR of
both \SpecL{} and C programs.
We will continue to refer to these IRs
as \SpecL{} and C respectively.

\subsection{Equivalence Definition}
\label{sec:eqdef}
Given (1) a \SpecL{} program specification $S$, (2) a C implementation
$C$,
(3) a precondition $Pre$ that
relates the initial inputs {\tt Input}$_S$
and {\tt Input}$_C$ to
$S$ and $C$ respectively, and (4) a postcondition
$Post$ that relates the final outputs {\tt Output}$_S$
and {\tt Output}$_C$ of $S$ and $C$ respectively\footnote{{Input}$_C$ and {Output}$_C$ include the initial and final memory state of $C$ respectively.}:\\
$S$ and $C$ are equivalent under precondition $Pre$ if
for all possible inputs {\tt Input}$_S$
and {\tt Input}$_C$,
such that $Pre(\mathrm{\tt Input}_S, \mathrm{\tt Input}_C)$
holds,
$S$'s execution is
well-defined on {\tt Input}$_S$, {\em and} C's
memory allocation requests during its execution on
{\tt Input$_C$} are successful,
then both programs $S$ and $C$ produce outputs
that satisfy $Post$.
$$
(Pre(\mathrm{\tt Input}_S, \mathrm{\tt Input}_C) \land (S\ \  \mathrm{\tt def}) \land (C\ \ \mathrm{\tt fits})) \Rightarrow Post(\mathrm{\tt Output}_S, \mathrm{\tt Output}_C)
$$

The $(S\ \  \mathrm{\tt def})$ antecedant states that we are only
interested in proving equivalence for well-defined
executions of $S$, i.e., executions that are
free of undefined
behaviour (UB). For example, division-by-zero is UB
in $S$.
Sometimes
the user may be interested in constraining the nature of
inputs to the C program, e.g., the {\tt strlen(char* s$_C$)} function
is well-defined only if {\tt s$_C$} is not null.
Thus,
for {\tt strlen}, we are only interested in computing
equivalence for non-null input pointers. \SpecL{} has
no notion of pointers and so this condition cannot be encoded
in $S$ alone.
In these cases, we use a combination of
$Pre$ and $(S\ \  \mathrm{\tt def})$
to constrain the executions of $C$ for which we
are interested in proving equivalence. In the {\tt strlen} example,
$(S\ \  \mathrm{\tt def})$
is encoded as an
abstract condition that the input string {\tt s}$_S$ is ``not invalid'',
written $\neg$({\tt s$_S$ {\em is} SInvalid}), where {\tt SInvalid} is
a constructor for the \SpecL{} {\tt String} type.
The precondition $Pre$ then contains the relation {\tt (s$_S$ {\em is} SInvalid)$\Leftrightarrow$({\tt s$_C$}=0)}.
This ensures that we compute equivalence only for those executions
of $C$ where the input pointer {\tt s$_C$} is non-null.
The use of an explicit constructor for expressing ill-formedness
of \SpecL{} input values along with a cleverly chosen $Pre$ allows us
to constrain the executions to $Spec$ and $C$ to well-formed inputs
only during equivalence check.
Please refer to {\tt Chapter \ThesisChapterEval{}} of thesis for
a more detailed explaination of this strategy for
the function {\tt strchr}.

The $(C\ \  \mathrm{\tt fits})$ antecedant states that we
prove equivalence only if the C program's memory requirements
fit within the available system memory, i.e.,
only for those executions of $C$ in which all
memory allocation requests (through {\tt malloc} calls)
are successful.

The returned values of $S$ and
$C$ procedures form their observable outputs.
For $S$, the returned values are
explicit and may include well-typed
recursive data-structure objects.
For $C$,
observable returned values also include portions
of the implicit memory state at
program exit. The postcondition relates these outputs of
the two programs.
%, potentially through \recursiveRelations{}.
%, potentially through the use of
%\recursiveInvariant{} shapes.

%Our proof search procedure involves the automatic inference
%of invariants relating the intermediate values of the two
%programs at correlated program points.
%As we will
%see through examples, these intermediate relations
%can often be derived from the
%relations between the programs' inputs ($Pre$)
%or between the programs' outputs ($Post$) supplied
%by the user.