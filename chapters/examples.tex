\section{Algorithm through Linked List Examples}
A {\tt List} ADT in the
\SpecL{} program is defined at line {\tt A0}
in \cref{fig:llAllocSpec}. An empty list is represented
by the
constant {\tt LNil()}\footnote{{\tt LNil()} represents the application
of the nullary constructor {\tt LNil} on the unit value {\tt ()}. For brevity,
we will simply write {\tt LNil} for {\tt LNil()} henceforth.};
a non-empty list uses
the {\tt LCons} constructor to combine its first value ({\tt val:i32})
and the remaining list ({\tt tail:List}).
\SpecL{} supports
{\tt i<N>} (bitvectors of length {\tt N}),
{\tt bool}, and {\tt unit} types, also called {\em scalar types}.
\SpecL{}'s type system prevents
the creation of cycles in ADT values.
If {\tt l} is
an object of type {\tt List}, then to access its
constituent values, we may expand (or unroll) {\tt l} to
\begin{small}
\begin{equation}\label{eqn:specDeconstruct}
U_S: {\tt l\ =\ \underline{if}\ l}\ \ is\ \ {\tt LNil\ \ \underline{then}\ \ LNil\ \ \underline{else}\ \ LCons(l.val,l.tail)}
\end{equation}
\end{small}
In this expanded representation of {\tt l},
the {\em sum-deconstruction} operator\footnote{The sum-deconstruction operator `\sumDtor{}' for a sum type $T$ must contain exactly one branch for each top-level value constructor of $T$.
For example, `\sumDtor{}' for the {\tt List} type must have exactly two branches of the form {\tt LNil} and {\tt LCons}($e_1$,$e_2$) for some expressions $e_1$ and $e_2$.}
`\sumDtor{}'
deconstructs a sum type where the
\underline{if} condition `{\tt l} {\em is} {\tt Constructor}'
checks whether the top level constructor of {\tt l} is `{\tt Constructor}'.
If {\tt l}
is a non-empty list constructed through {\tt LCons},
then {\tt l.val} and {\tt l.tail}
are used to access {\tt l}'s first value and {\tt l}'s tail respectively.
The right-hand side of
\cref{eqn:specDeconstruct} can also be viewed as an executable
program that unrolls the input {\tt List} object {\tt l} once and
outputs a {\tt List} object constructed from {\tt l}'s constituents --- we
call \cref{eqn:specDeconstruct} the {\em unrolling procedure} $U_S$ of the {\tt List} ADT.
%We will later unify this expanded representation of a {\tt List} object with
%its C implementation.

\begin{figure}
\begin{tabular}{@{}c@{}c@{}}
\begin{subfigure}[b]{0.5\textwidth}
\begin{center}
{\includegraphics[scale=1.4]{chapters/figures/figMallocSpecCfg.pdf}}
\vspace{15pt}
\end{center}
\caption{\label{fig:llAllocSpecIRCFG}CFG of \SpecL{} Program}
\end{subfigure}%
&
\begin{subfigure}[b]{0.5\textwidth}
\begin{center}
{\includegraphics[scale=1.4]{chapters/figures/figMallocCCfg.pdf}}
\end{center}
\caption{\label{fig:llAllocCCFG}CFG of C Program}
\end{subfigure}%
\\
\end{tabular}
\caption{\label{fig:mallocSpecCFGAndCCFG}CFG representation of Spec and C IRs shown in \cref{fig:llAllocSpecIR,fig:llAllocCIR} for the {\tt mk\_list} procedures in \cref{fig:llAllocSpec,fig:llAllocC} respectively.}
\end{figure}


\Cref{fig:llAllocSpecIRCFG,fig:llAllocCCFG} show the Control-Flow Graph (CFG) representations
of the \SpecL{} and C programs in \cref{fig:llAllocSpecIR,fig:llAllocCIR} respectively.
The CFG nodes represent PC locations of the program, and edges represent
transitions through instruction execution. For brevity, we sometimes represent
multiple program instructions with a single edge, e.g., in \cref{fig:llAllocCCFG}, the edge {\small \tt C5$\rightarrow$C3}
represents the path {\small \tt C5$\rightarrow$C6$\rightarrow$C7$\rightarrow$C8$\rightarrow$C3}. A control-flow edge is associated
with an {\em edge condition} (the condition under which that edge is taken),
a {\em transfer function} (how the program state is mutated if that edge is taken),
and a {\em UB assumption} (what condition should be true for the program
execution to be
well-defined across this edge). For example, the UB assumption associated with a division
instruction in $S$ will encode that the divisor must be non-zero.

\subsection{Product CFG}
\label{sec:productCFG}
We construct
a {\em bisimulation relation} to identify
equivalence between two programs. 
A bisimulation relation correlates
the transitions of $S$ and $C$ in lockstep, such that this
lockstep execution ensures
identical observable behavior.
An equivalence proof through bisimulation construction
can be represented using a
{\em product program} \cite{covac} and the CFG of a product program is called a {\em product-CFG}.
\Cref{fig:llAllocProductCFG} shows the CFG of the product program, also called
the product-CFG,
that encodes the lockstep execution (bisimulation relation) between the
CFGs in \cref{fig:llAllocSpecIRCFG} and \cref{fig:llAllocCCFG}. 

\begin{figure}[H]
\begin{tabular}{cc}
\begin{subfigure}[b]{1\textwidth}
\begin{center}
{\includegraphics[scale=1.2]{chapters/figures/figMallocProductCfg.pdf}}
\end{center}
\end{subfigure}%
\end{tabular}
\caption{\label{fig:llAllocProductCFG}Product-CFG between the CFGs in \cref{fig:llAllocSpecIRCFG,fig:llAllocCCFG}}
\end{figure}


A node in the product-CFG is formed by pairing nodes
of $S$ and $C$ CFGs, e.g., {\tt (S3:C5)} is formed by
pairing {\tt S3} and {\tt C5}.
If the lockstep
execution is at node {\tt (S3:C5)} in the product-CFG,
then $S$'s execution is at {\tt S3} and $A$'s execution
is at {\tt C5}.
The start node {\tt (S0:C0)} of the product-CFG
correlates the start nodes of the CFGs of both programs.
Similarly, the exit node {\tt (SE:CE)}
correlates the exit nodes of the CFGs of both programs.

An edge in the product-CFG is formed by pairing a path (a sequence of edges) in $S$ with a path in $C$.
A product-CFG edge encodes the lockstep execution of correlated
transitions (or paths).
For example, the product-CFG edge {\tt (S3:C5)}$\rightarrow${\tt (S3:C3)} is formed by pairing the
{\small \tt (S3$\rightarrow$S5$\rightarrow$S3)} and {\small \tt C5$\rightarrow$C3}
in \cref{fig:llAllocSpecIRCFG,fig:llAllocCCFG},
and represents that when $S$ makes a transition {\small \tt (S3$\rightarrow$S5$\rightarrow$S3)},
then $C$ makes the transition {\small \tt C5$\rightarrow$C3}
in lockstep.
The edge {\small \tt (S3:C3)$\rightarrow$(S3:C4)}
correlates the $\epsilon$ path (no transition) in $S$ with {\small \tt C3$\rightarrow$C4} in $C$.
In general, a product-CFG edge $e$ may correlate a finite path $\rho_S$
in $S$ with a finite path $\rho_C$ in $C$,
written $e=(\rho_S,\rho_C)$.

\begin{table}[]
\begin{center}
\caption{\label{tab:llproductInv}Node Invariants for Product-CFG in \cref{fig:llAllocProductCFG}}
\begin{footnotesize}
\begin{tabular}{|c|llll|}
\hline
{\bf PC-Pair} & \multicolumn{4}{c|} {\bf Invariants} \\
\hline
\hline
(\scpc{0}{0}) &
\multicolumn{4}{l|} {\Tstrut $\circled{P}\ \sv{n}=\cv{n}$} \\
(\scpc{3}{3}) &
\Tstrut  $\circled{\scriptsize I1}\  \sv{n}=\cv{n}$ & $\circled{\scriptsize I2}\  \sv{i}=\cv{i}$ & $\circled{\scriptsize I3}\  \sv{i} \leq_{u} \sv{n}$ & $\circled{\scriptsize I4}\  \sv{l} \indEq{} \lifted{list}{\mem{}}{lnode}{\cv{l}}$ \\
(\scpc{3}{4}) (\scpc{3}{5}) &
\Tstrut  $\circled{\scriptsize I5}\  \sv{n}=\cv{n}$ & $\circled{\scriptsize I6}\  \sv{i}=\cv{i}$ & $\circled{\scriptsize I7}\  \sv{i} <_{u} \sv{n}$ & $\circled{\scriptsize I8}\  \sv{l} \indEq{} \lifted{list}{\mem{}}{lnode}{\cv{l}}$ \\
(\scpc{E}{E}) &
\multicolumn{4}{l|} {\Tstrut  $\circled{E}\  \sv{ret} \indEq{} \lifted{list}{\mem{}}{lnode}{\cv{ret}}$} \\
\hline
\end{tabular}   
\end{footnotesize}
\end{center}
\end{table}


At the start node {\tt (S0:C0)} of the product-CFG, the precondition $Pre$
(labeled \circled{\small P})
ensures the equality
of input arguments ${\tt n_{S}}$ and ${\tt n_{C}}$ at programs' entry.
Inductive invariants are inferred at each
product-CFG node that
relate the variables of $S$ with variables
and memory locations of $C$.
The inductive invariants
are identified by running an invariant inference
algorithm on the product-CFG. The inductive invariants
for our example are shown in \cref{tab:llproductInv}.
For example, at node {\tt (S3:C5)} in \cref{fig:llAllocProductCFG},
${\tt i}_S={\tt i}_C$ is an inductive
invariant. If the inferred invariants ensure
that the postcondition $Post$ holds at the
exit node {\tt (SE:CE)} (labeled \circled{\small E}), we have shown
equivalence of both programs.

\subsection{Recursive relations}
In \cref{tab:llproductInv}, the relation between
programs' variables at product-CFG nodes {\tt S3:C3}, {\tt S3:C4} and {\tt S3:C5}
is encoded as a \recursiveRelation{}:
``{\tt l}$_{S}\indEq{}${\tt Clist}$^{{\tt lnode}}_{m}$({\tt l}$_C$)'' where {\tt l}$_{S}$ and {\tt l}$_C$ represent the {\tt l} variables in the \SpecL{} and
C programs respectively, {\tt lnode} represents the C {\tt struct} type that contains
the {\tt value} and {\tt next} fields, and
$m$ represents a byte-addressable array representing
the current memory state of
the C program.
$l_1\indEq{}l_2$ is read {\em $l_1$ is recursively equal to
$l_2$}, i.e., $l_1$ and $l_2$ are isomorphic and
have equal
values\footnote{$l_1\indEq{}l_2$ and
$l_1=l_2$ are equivalent --- the former emphasizes
the recursive nature of the values being compared.}.
The {\em lifting constructor} {\tt Clist}$^{{\tt lnode}}_m({\tt p})$ is a
constructor that {\em lifts}
the C pointer value {\tt p} (pointing to
an object of {\tt struct lnode}) and the C
memory state $m$ to a \SpecL{} {\tt List} value.
{\tt Clist}$^{{\tt lnode}}_m({\tt p})$ is defined through its
unrolling procedure as:
\begin{small}
\begin{equation}\label{eqn:clist}
\begin{split}
U_C: \mathrm{\tt Clist}^{{\tt lnode}}_m&\mathrm{\tt (p:i32)} = \mathrm{\tt \underline{if}\ (p==0)}\ \mathrm{\tt \underline{then}\ }{\tt LNil} \\
                                                         & \mathrm{\tt \underline{else}\ {\tt LCons}(}\structPointer{p}{m}{{\tt lnode}}{val},\ {\tt Clist}_m^{\tt lnode}(\structPointer{p}{m}{{\tt lnode}}{next}))
\end{split}
\end{equation}
\end{small}
By construction, this unrolling procedure $U_C$ is
isomorphic to {\tt List}'s  unrolling procedure $U_S$ in \cref{eqn:specDeconstruct}.
``{\small $\structPointer{p}{m}{s}{f}$}'' represents the
field `{\tt f}' of the `struct {\tt s}' object pointed-to by pointer `{\tt p}'
in memory
state `$m$'.  When represented
in C-like syntax, `{\small $\structPointer{p}{m}{s}{f}$}'
is equivalent to
``{\small {\tt *((typeof s.f)*)(\&m[p+offsetof(s,f)])}}'',
%.\footnote{``{\tt offsetof(s, f)}'' represents the offset in an object of type `{\tt struct s}' where field {\tt f} lives. ``{\tt typeof s.f}'' represents the type of field {\tt f} in `{\tt struct s}'.}}\\
i.e., the expression `{\small $\structPointer{p}{m}{s}{f}$}'
returns the bytes in the memory array `$m$' starting
at address `{\small {\tt p+offsetof(s,f)}}' and interpreted as an object of
type `{\small \tt typeof s.f}'.

Note the recursive nature of the lifting value constructor {\tt Clist}: if the pointer {\tt p} of type {\tt i32}\footnote{The IR lowers integers and pointers in C to bitvectors of type {\tt i<N>}. e.g., {\tt i32} is a 32-bit bitvector type.} is zero, then this represents the empty list ({\tt LNil}); otherwise it represents the list formed by {\tt LCons}-ing
the value stored at {\tt p->value} in memory $m$ and the list formed by recursively lifting {\tt p->next} using
{\tt Clist} in memory $m$.  The recursive lifting
constructor {\tt Clist} allows us to compare C values and \SpecL{} values for equality; a
relation involving a recursive constructor is called a {\em \recursiveRelation{}}.

We later discuss in \cref{sec:searchAlgoFormal}
how a product-CFG can be constructed automatically through a counterexample-guided
search. Before that, we discuss the proof obligations that arise from a given
product-CFG. Consider the
product-CFG in \cref{fig:llAllocProductCFG}. Assuming that the precondition \circled{\small P} holds at the entry node
{\tt S0:C0} of this product-CFG, a bisimulation
check involves checking that the invariants at the other
product-CFG nodes hold too, and consequently the postcondition \circled{\small E}
holds at the exit node {\tt SE:CE}.
Recall that the precondition \circled{\small P} and the postcondition \circled{\small E}
are provided by the user, but all the other invariants are inferred automatically.

\subsection{Proof Obligations}
\label{sec:proofObligations}
We use relational Hoare triples to express these proof obligations \cite{relationalHoareLogic,hoareTriple}.
If $\phi$ denotes a predicate relating the machine states
of programs $S$ and $C$, then for a product-CFG edge $e=(\rho_S,\rho_C)$,
$\{\phi_s\} (e) \{\phi_d\}$ denotes the condition:
if the machine states $\sigma_S$ and $\sigma_C$ of programs $S$
and $C$ are related through precondition $\phi_s(\sigma_S,\sigma_C)$ and
paths $\rho_S$ and $\rho_C$ are executed in $S$
and $C$ respectively (implying the path conditions
hold), then execution terminates normally in states $\sigma_S^{'}$ (for $S$)
and $\sigma_C^{'}$ (for $C$) where
postcondition $\phi_d(\sigma_S^{'}, \sigma_C^{'})$ hold.
$\{\phi_s\} (e) \{\phi_d\}$ can also be written as
$\{\phi_s\} (\rho_S,\rho_C) \{\phi_d\}$.

For every product-CFG edge $e = (s\rightarrow d) = (\rho_S,\rho_C)$
in \cref{fig:llAllocProductCFG}, we thus
need to prove
$\{\phi_s\} (\rho_S,\rho_C) \{\phi_d\}$, where $\phi_s$ and $\phi_d$
are the node invariants
(shown
in \cref{tab:llproductInv})
at nodes $s$ and $d$ of the product-CFG respectively.
%Each set of paths (or pathset) $\xi$ is composed of multiple
%individual paths, say $\rho_1,\rho_2,...\rho_n$.
%The proof obligation, $P = (\{\phi_s\} \xi_S,\xi_C \{\phi_d\})$,
%can therefore be broken down into a sequence of
%smaller proof obligations of the form,
%$P_{ij}=(\{\phi_s\} \rho^i_S,\rho^j_C \{\phi_d\})$, where $\rho^i_S$
%and $\rho^j_C$ represent the $i$th and $j$th paths in
%pathsets $\xi_S$ and $\xi_C$ respsectively. If pathsets
%$\xi_S$ and
%$\xi_C$ contain
%$n_S$ and $n_C$ individual paths respectively, then we
%create $n_S*n_C$ smaller proof obligations $P_{ij}$ (cartesian product)
%during the discharge of $P$.
%$P$ holds iff all $P_{ij}$s hold; a counterexample that
%falsifies any $P_{ij}$ would also falsify $P$. As we will
%see later, this
%breaking down of $P$
The weakest-precondition transformer
is used to translate a Hoare triple $\{\phi_s\} (\rho_S,\rho_C) \{\phi_d\}$
to the following first-order logic formula:
\begin{equation}\label{eqn:firstOrderFormula}
(\phi_s \land pathcond_{\rho_S} \land pathcond_{\rho_C} \land ubfree_{\rho_S}) \Rightarrow {\tt WP}_{{\rho_S,\rho_C}}(\phi_d)
\end{equation}
Here, $pathcond_{\rho_X}$ represents the condition that
path $\rho_X$ is taken in program $X$.
$ubfree_{\rho_S}$ represents the condition that
the execution of program $S$ along path $\rho_S$ is
free of undefined behavior.
${\tt WP}_{{\rho_S,\rho_C}}(\phi_d)$ represents the weakest-precondition
of the predicate $\phi_d$ across the product-CFG edge $e = (\rho_S,\rho_C)$.
We will use ``{\tt LHS}'' and ``{\tt RHS}'' to
refer to the left and right hand sides of the
implication operator ``$\Rightarrow$'' in \cref{eqn:firstOrderFormula}.

\subsection{Proof Discharge Algorithm and Its Soundness}
\label{sec:proofDischarge}
We call an algorithm that evaluates the
truth value of a proof obligation, a {\em proof discharge algorithm}.
A proof discharge algorithm is {\em precise} if for all
proof obligations, the
truth value evaluated by the algorithm is identical to the
proof obligation's
{\em actual} truth value.
A proof discharge algorithm is {\em sound} if:
(a) Whenever it evaluates a proof obligation to true, the
actual truth value of that proof obligation is also true;
(b) Whenever it generates a counterexample, that counterexample
must falsify the proof obligation.
% \begin{itemize}
% \item Whenever it evaluates a proof obligation to true, the
% actual truth value of that proof obligation is also true.
% \item Whenever it generates a counterexample, that counterexample
% must falsify the proof obligation.
% \end{itemize}
However, it is possible for a sound proof discharge
algorithm to return false (without a counterexample)
when the proof obligation was actually true.

For
the proof obligations generated by our equivalence
procedure,
it is always safe for a proof discharge algorithm to
return false (without a counterexample).
If a proof
discharge algorithm conservatively
evaluates a proof obligation to false (when it
was actually true), it may
prevent the overall equivalence proof from completing successfully;
however, importantly, the overall equivalence procedure remains sound.

Resolving the truth value of a proof obligation that contains
a \recursiveRelation{} such as ${\tt l_{S}\indEq{}Clist^{{\tt lnode}}_{m}(l_{C})}$ is unclear.
Fortunately, the shapes of the proof obligations generated
by our equivalence checking algorithm are restricted, which
makes it possible to soundly resolve these proof obligations.

Our equivalence
checking algorithm ensures that, for an invariant
$\phi_s=({\tt \phi^{1}_s \land \phi^{2}_s \land ... \land \phi^{k}_s})$,
at any node $s$
of a product-CFG,
if a \recursiveRelation{} appears in $\phi_s$, it
must be one of $\phi^{1}_s$, $\phi^{2}_s$, ..., or $\phi^{k}_s$. We call
this the {\em conjunctive \recursiveRelation{}} property of an invariant $\phi_s$.

A proof obligation
$\{\phi_s\}(e)\{\phi_d\}$, where $e=(\rho_S,\rho_C)$,
gets lowered using
${\tt WP}_{e}(\phi_d)$ (as shown in \cref{eqn:firstOrderFormula}) to a first-order logic formula of the following form:
\begin{equation}\label{eqn:proofObligationShape}
({\tt \eta^{l}_1 \land \eta^{l}_2 \land ... \land \eta^{l}_m}) \Rightarrow ({\tt \eta^{r}_1 \land \eta^{r}_2 \land ... \land \eta^{r}_n})
\end{equation}
In
this formula,
the {\tt LHS} and {\tt RHS} are
written as conjunctions of $\eta^{l}_i$ and $\eta^{r}_j$ respectively (for $1\leq i\leq m$, $1\leq j\leq n$).
Each $\eta^{r}_j$ relation is obtained from
${\tt WP}_{e}(\phi^{j}_d)$, where $\phi_d=({\tt \phi^{1}_d \land \phi^{2}_d \land ... \land \phi^{n}_d})$.
Thus, due to the conjunctive \recursiveRelation{} property of $\phi_s$ and $\phi_d$, any
\recursiveRelation{} in \cref{eqn:proofObligationShape} must appear as
one of $\eta^{l}_i$ or $\eta^{r}_j$.

To simplify proof obligation discharge,
we break a first-order logic proof obligation $P$ of the form in \cref{eqn:proofObligationShape}
into multiple smaller proof obligations
of the form
$P_j:({\tt LHS}\Rightarrow{\tt \eta^{r}_j})$, for $j=1..n$. Each proof obligation
$P_j$ is then discharged separately.  We call this conversion from
a bigger query to multiple smaller queries, {\em RHS-breaking}.

%We say that two relations $\eta_1$ and $\eta_2$ {\em interfere}, written
%$\eta_1 \interfere{} \eta_2$,
%iff the intersection
%of the sets of free variables mentioned in $\eta_1$ and $\eta_2$
%is non-empty.  For example, $a\indEq{}b$ and $c\indEq{}d$
%do not interfere, but $a\indEq{}b$ and $c\indEq{}{\tt LCons}(0,b)$ interfere.
%
%In a proof obligation of the form in \cref{eqn:proofObligationShape},
%two relations $\eta_1$ and $\eta_2$ {\em transitively interfere} iff
%there exists a relation $\eta_3$ in the proof
%obligation such that $\eta_1\interfere{}\eta_3$ and $\eta_3\interfere{}\eta_2$.
%This notion of transitive interference can be lifted
%to Hoare triples: in a proof obligation $\{\phi_s\}(e)\{\phi_d\}$,
%where $\phi_{s}=\phi^{1}_s\land...\phi^{k}_{s}$ and $\phi_{d}=\phi^{1}_d\land...\phi^{n}_{d}$,
%an {\tt LHS} conjunctive clause $\phi^{i}_s$ transitively
%interferes with an {\tt RHS} conjunctive clause $\phi^{j}_d$ iff
%$\phi^{i}_s$
%transitively interferes with ${\tt WP}_{e}(\phi^{j}_d)$ in
%the lowered first-order logic formula.

%We provide a sound (but imprecise) proof discharge
%algorithm that converts
%a proof obligation generated by our equivalence
%checker into a series of SMT queries.
%Our
%algorithm begins
%by categorizing a proof obligation
%into three categories, and each
%category is discussed separately
%in subsequent sections.  These categories are created
%based on an iterative unrolling and unification
%procedure, which we describe next.
%We use an unroll parameter $k$ for
%our categorization.

\subsection{Iterative Unification and Unrolling}
\label{sec:unifyAndUnroll}

We begin with some definitions.
An expression $e$ whose top-level constructor
is a lifting constructor, e.g., $e={\tt Clist^{{\tt lnode}}_m({\tt l}_C)}$,
is called a {\em lifted expression}.
An expression $e$ of the form {\tt v.a$_1$.a$_2$...a$_n$} i.e. a variable
with {\em zero} or more {\em product deconstruction} operators applied on it, is called a {\em pseudo-variable}.
An expression $e$ in which (1) all product deconstructors (e.g. {\tt `.tail'}) appear as part
of a {\em pseudo-variable} and (2) each {\em sum-is} operator (e.g. {\tt `is LCons'})
operate on a {\em pseudo-variable}, is called a {\em canonical expression}.

Consider the expression tree of a canonical expression $e$ of ADT $T$,
formed using the ADT value constructors and the `\underline{if}-\underline{then}-\underline{else}'
sum-deconstruction operator.
% Consider an expression tree $e$ of ADT $T$\footnote{XXX:can contain value constructors of other ADT types as well}
% formed using the value constructors
% of $T$ and the `\underline{if}-\underline{then}-\underline{else}'
% sum-deconstruction operator.
The leaves of $e$ (also called atoms of $e$) are
the pseudo-variables (of scalar or ADT type),
the scalar expressions (of {\tt unit}, {\tt bool}, or {\tt i<N>} types),
and lifted expressions.

The
{\em expression path} to a node $v$ in $e$'s tree
is the path from the root of $e$
to that node $v$. The {\em expression path condition} represents
the conjunction of all the \underline{if} conditions (if
the \underline{then} branch is taken on the expression path), or their
negation (if the \underline{else} branch is taken on the
expression path) seen on the expression path.  For example,
in expression {\tt \underline{if} (c) \underline{then} a \underline{else} b},
the expression path condition of {\tt c} is {\tt true}, of {\tt a} is {\tt c},
and of {\tt b} is $\neg{\tt c}$.

When we attempt to unify two expressions, we unify the
structures created by the value constructors and the `\underline{if}-\underline{then}-\underline{else}'
operator of their canonical forms.
% When we attempt to unify two expressions, we unify the
% structures created by the value constructors and
% the \underline{if}-\underline{then}-\underline{else}
% operator of a \SpecL{} ADT.
The unification procedure either fails to unify, or it
returns tuples $(p_1,p_2,a_1,e_2)$ where atom $a_1$
at expression path condition $p_1$ in one expression is correlated with
expression $e_2$ at expression path condition $p_2$ in the other expression.

For two non-atomic expressions $e_1$
and $e_2$ to unify successfully, it must be true that
either the top-level
node in both $e_1$ and $e_2$ have the same
value constructor (in which case a unification is attempted
for each of the children of the top-level constructor),
{\em or} the top-level node in one of $e_1$ or $e_2$
is \underline{if}-\underline{then}-\underline{else.} If
the top-level node of $e_1$ is ``\underline{if} (c) \underline{then} $e^{\tt then}_1$ \underline{else} $e^{\tt else}_1$,
we attempt to unify both $e^{\tt then}_1$ and $e^{\tt else}_1$ with $e_2$ and
return success iff any of these attempts succeed (similarly for $e_2$).
Whenever we descend down an \underline{if}-\underline{then}-\underline{else} operator,
we conjunct the corresponding \underline{if} condition (for $e_1^{\tt then}$) or
its negation (for $e_1^{\tt else}$) to the respective
expression path condition. If one of
$e_1$ and $e_2$ (say $e_2$) is atomic, unification always succeeds and
returns $(p_2,p_1,e_2,e_1)$.

With each atom of an ADT type, we associate a corresponding unrolling procedure.
For example, the unrolling procedure for a \SpecL{} variable of {\tt List}
type is $U_S$ (\cref{eqn:specDeconstruct}),
whereas the unrolling procedure for a lifted expression
of type {\tt List} is $U_C$ (\cref{eqn:clist}).

Given two expressions $e_a$ and $e_b$
of an ADT $T$ at expression path conditions $p_a$ and $p_b$
respectively, an {\em iterative
unrolling and unification procedure} $\Theta(e_a,e_b,p_a,p_b)$ is used
to identify a set of correlation tuples between the
atoms in the two expressions.
This
iterative procedure
proceeds by attempting
to unify $e_a$ and $e_b$. If this unification
fails, we return a unification failure
for the original expressions $e_a$ and $e_b$. Else, we obtain
correlation between atoms and
expressions (with their expression path conditions). If the unification
correlates an atom $a_1$ at expression path condition $p_1$
with another atom $a_2$ at expression path condition $p_2$, we add $(p_1,a_1,p_2,a_2)$
to the final output.  If the unification correlates
an atom $a_1$ at expression
path condition $p_1$ to a non-atomic expression $e_2$ at
expression path condition $p_2$,
we unroll $a_1$ once using its unrolling procedure to
obtain expression $e_1$. The unification algorithm
then proceeds by unifying $e_1$ and $e_2$ through
a recursive call to $\Theta(e_1,e_2,p_a\land{}p_1,p_b\land{}p_2)$.

The maximum number of unrollings performed by
$\Theta(e_a,e_b,p_a,p_b)$ (before converging)
is upper bounded by the sum of number of $T$ value
constructors in $e_a$ and $e_b$.

If a proof obligation involves a \recursiveRelation{}
$e_a\sim{}e_b$, we unify $e_a$ and $e_b$
through a call to $\Theta(e_a, e_b, {\tt true}, {\tt true})$.
For example,
the unification of ``{\tt \underline{if} (c$_1$) \underline{then} LNil \underline{else} LCons(0, l$_S$)}'' and
``{\tt \underline{if} (c$_2$) \underline{then} LNil \underline{else} LCons(0, {\tt Clist$^{{\tt lnode}}_m({\tt l}_C)$})}''
yields the correlation
tuples: {\tt (c$_1$,(),c$_2$,())}, {\tt ($\neg$c$_1$,0,$\neg$c$_2$,0)} and {\tt ($\neg$c$_1$,l$_S$,$\neg$c$_2$,$Clist^{{\tt lnode}}_m({\tt l}_C)$)}.

If the set of $n$ tuples obtained
after a successful unification
of $e_a\sim{}e_b$
are $(p^i_1, a^i_1, p^i_2, a^i_2)$ (for $i=1\ldots{}n$), then
$e_a\indEq{}e_b \Leftrightarrow{} \bigwedge\limits_{i=1}^n((p^i_1=p^i_2) \land ((p^i_1\land{}p^i_2)\rightarrow{}(a^i_1=a^i_2)))$\footnote{If $a^i_1$ and $a^i_2$ are recursive data structures, then we replace $a^i_1=a^i_2$ with $a^i_1\indEq{}a^i_2$.}.
We call
$\bigwedge\limits_{i=1}^n((p^i_1=p^i_2) \land ((p^i_1\land{}p^i_2)\rightarrow{}(a^i_1=a^i_2)))$
the {\em decomposition}
of $e_a\indEq{}e_b$.
Each conjunctive clause of one of the forms $(p^i_1=p^i_2)$ and $((p^i_1\land{}p^i_2)\rightarrow{}(a^i_1=a^i_2))$ in
this decomposition is called a {\em decomposition clause}.
A decomposition clause may relate only atomic values, i.e.,
in the decomposed form, all recursive relations
relate only ADT variables and/or lifted expressions.
The decomposition for a failed unification is defined as {\tt false}.
We {\em decompose}
a \recursiveRelation{} by replacing it with its
decomposition.
We {\em decompose} a proof obligation $P$ by
decomposing all recursive relations in $P$.
%Further, we 

\subsection{$k$-unrolling with respect to an unrolling procedure}
We can {\em unroll an expression $e$ with respect to an unrolling
procedure $U$}
by substituting all occurrences
of the constructor ({\tt LHS} in $U$) by its
unrolled version ({\tt RHS} in $U$) and decomposing it.
For example, an unrolling
of ${\tt LHS}\Rightarrow{}{\tt l_S\indEq{}Clist^{{\tt lnode}}_{m}(0)}$
with respect to $U_C$ (\cref{eqn:clist})
yields
$P: {\tt LHS}\Rightarrow{}{\tt l_S\indEq{}LNil}$.
Unrolling $P$ with respect to $U_S$ (\cref{eqn:specDeconstruct})
yields
${\tt LHS}\Rightarrow{}{\tt l_S\ \ is\ \ LNil}$.

A $k$-unrolling of an expression $e$ w.r.t. unrolling procedure $U$
is obtained by unrolling a $(k-1)$ unrolling of $e$ w.r.t. $U$.
We say an expression $e$ is $k$-unrolled (without specifying
the unrolling procedure $U$), we imply that
$e$ is $k$-unrolled with respect to the unrolling procedures of
{\em all} the lifting constructors appearing in $e$.

For a first-order
logic proof obligation
$P: {\tt LHS}\Rightarrow{\tt RHS}$, we identify
a $k$-unrolling of $P$ (for a fixed unrolling parameter $k$).
After unrolling,
we eliminate those decomposition clauses
$(p^i_1\land{}p^i_2)\rightarrow{}(a^i_1=a^i_2)$ whose
path condition $(p^i_1\land{}p^i_2)$ evaluates to false
under the non-recursive conjunctive clauses in the {\tt LHS}. We categorize the proof obligation based
on this $k$-unrolled form of $P$.

\subsection{Category 1: The $k$-unrolled form of $P$
contains no recursive relations}
\label{sec:cat1}
In \cref{fig:llAllocProductCFG}, consider a proof obligation generated
across the product-CFG edge ${\tt (S0:C0) \rightarrow (S3:C3)}$
while checking if the ${\small \circled{I4}}$ invariant, $l_{S}\indEq{}Clist^{{\tt lnode}}_{m}(l_{C})$,
holds at {\tt (S3:C3)}:
$\{\phi_{{\tt S0:C0}}\} ({\tt S0\rightarrow S3}, {\tt C0\rightarrow C3}) \{l_{S}\indEq{}Clist^{{\tt lnode}}_{m}(l_{C})\}$.
The precondition $\phi_{{\tt S0:C0}}=(n_S=n_C)$ does not contain
a \recursiveRelation{}.
When lowered to first-order logic through {\tt WP$_{{\tt S0\rightarrow S3},{\tt C0\rightarrow C3}}$}, this translates to
``$(n_S=n_C)\Rightarrow({\tt LNil}\indEq{}{\tt Clist^{{\tt lnode}}_{m}(0)})$''.
Here, {\tt LNil} is obtained for {\tt l$_{S}$} and {\tt 0} (or NULL) is
obtained for {\tt l$_C$}.
The one-unrolled form of this proof obligation yields
$(n_S=n_C)\Rightarrow{}{\tt LNil}\indEq{}{\tt LNil}$ which trivially resolves to true.

Consider another example of a proof obligation,
$\{\phi_{{\tt S0:C0}}\} ({\tt S0\rightarrow S3\rightarrow S5\rightarrow S3}, \\{\tt C0\rightarrow C3}) \{l_{S}\indEq{}Clist^{{\tt lnode}}_{m}(l_{C})\}$.
Notice, we have changed the path in $S$ to ${\tt S0\rightarrow S3\rightarrow S5\rightarrow S3}$ here.
In this case, the corresponding first-order logic condition evaluates
to: ``$(n_S=n_C)\Rightarrow({\tt LCons(0, LNil)}\indEq{}{\tt Clist^{{\tt lnode}}_{m}(0)})$''.
The one-unrolled form of this proof obligation converts
${\tt Clist^{{\tt lnode}}_{m}(0)}$ to {\tt LNil}, and
decomposes {\tt RHS} into {\tt false} (due to failed unification).
The proof obligation is further discharged using an SMT solver
which provides a counterexample (model) that evaluates the
formula to false. For example, the counterexample {\tt \{ n$_S$ $\mapsto$ 42, n$_C$ $\mapsto$ 42 \} }
evaluates this formula to false.
% because
% this fails to unify (during decomposition), it evaluates to false.
% When a
% proof obligation evaluates to false, the SMT solver
% provides a counterexample (model) that evaluates the condition
% to false.  For example, the counterexample {\tt n$_S$=n$_C$=42}
% evaluates this condition to false.
These counterexamples
assist in faster convergence
of our invariant inference and correlation search procedures (as we will discuss later).

Thus, we unify the structure and
values of the \SpecL{} objects on both sides of
the $\sim$ operator (after $k$ unrollings), and discharge the resulting
proof obligations (that relate bitvector and array values) using an SMT solver.
A more detailed workings of the procedure for discharging SMT proof obligations and handling counter-examples is given
in \cref{sec:smtdischarge}.
Assuming a capable enough SMT solver,
all proof obligations in Category 1 can be discharged precisely, i.e.,
we can always decide whether $P$ evaluates to true or false. If it
evaluates to false, we also obtain a counterexample.

\begin{figure}[htbp]
\begin{tabular}{cc}
\begin{subfigure}[b]{0.50\textwidth}
\begin{center}
\begin{allLangEnvScript}
~{\tiny \textcolor{mygray}{   }}~   
~{\tiny \textcolor{mygray}{S0:}}~ i32 sum_list (List l) {
~{\tiny \textcolor{mygray}{S1:}}~   i32 sum $\coloneq$ ${\tt 0_{i32}}$;
~{\tiny \textcolor{mygray}{S2:}}~   while $\mathrm{\tt\neg (l\ is\ LNil)}$:
~{\tiny \textcolor{mygray}{S3:}}~     sum $\coloneq$ sum + ${\tt l.val}$; // (l is LCons);
~{\tiny \textcolor{mygray}{S4:}}~     l   $\coloneq$ l.next;
~{\tiny \textcolor{mygray}{S5:}}~   return sum;
~{\tiny \textcolor{mygray}{SE:}}~ }
\end{allLangEnvScript}
\end{center}
\caption{\label{fig:llTraverseSpec}(Abstracted) Spec IR}
\end{subfigure}%
&
\begin{subfigure}[b]{0.50\textwidth}
\begin{center}
{\includegraphics[scale=0.9]{figSumListProductCfg.pdf}}
\end{center}
\caption{\label{fig:llTraverseProduct}Product-CFG}
\end{subfigure}%
\\
\begin{subfigure}[b]{0.50\textwidth}
\begin{center}
\begin{allLangEnvScript}
~{\tiny \textcolor{mygray}{\ \ \ }}~ unsigned sum_list (lnode* l);
~{\tiny \textcolor{mygray}{C0:}}~ i32 sum_list (i32 l) {
~{\tiny \textcolor{mygray}{C1:}}~   i32 sum $\coloneq$ ${\tt 0_{i32}}$;
~{\tiny \textcolor{mygray}{C2:}}~   while ${\tt l \neq 0_{i32}}$:
~{\tiny \textcolor{mygray}{C3:}}~     sum $\coloneq$ sum + $\structPointer{\tt l}{m}{\tt lnode}{val}$;
~{\tiny \textcolor{mygray}{C4:}}~     l   $\coloneq$ $\structPointer{\tt l}{m}{\tt lnode}{next}$;
~{\tiny \textcolor{mygray}{C5:}}~   return sum;
~{\tiny \textcolor{mygray}{CE:}}~ }
\end{allLangEnvScript}
\end{center}
\caption{\label{fig:llTraverseC}(Abstracted) C IR}
\end{subfigure}%
&
\begin{subfigure}[b]{0.50\textwidth}
\begin{center}
\begin{footnotesize}
\begin{tabular}{|c|l|}
\hline
\tt PC-Pair & \multicolumn{1}{c|} {\tt Invariants} \\
\hline
\hline
${\tt (S0:C0)}$ &
\Tstrut ${\tt {\circled{P}}\  l_{S}\indEq{}Clist^{lnode}_{m}(l_{C})}$ \\
\multirow{2}{*}{${\tt (S2:C2)}$} &
\Tstrut \BBstrut ${\tt {\scriptsize \circled{I1}}\  l_{S}\indEq{}Clist^{lnode}_{m}(l_{C})}$ \\ & ${\tt {\scriptsize \circled{I2}}\  sum_{S}=sum_{C}}$ \\
${\tt (SE:CE)}$ &
\Tstrut \BBstrut ${\tt {\circled{E}}\  ret_{S}=ret_{C}}$ \\
\hline
\end{tabular}
\end{footnotesize}
\vspace{13px}
\end{center}
\caption{\label{fig:llTraverseProductInv}Node Invariants of the Product-CFG}
\end{subfigure}%
\\
\end{tabular}
\caption{\label{fig:llTraverseSpecAndC}Traversing a list in Spec and C.}
\end{figure}


\subsection{Category 2: The $k$-unrolled form of $P$
contains a \recursiveRelation{} in the LHS but not in the RHS}
\label{sec:cat2}

Consider the pair of programs in \cref{fig:llTraverseSpec,fig:llTraverseC}
that traverse a list to compute the sum of all elements.
The corresponding product-CFG and its node
invariants that ensure observable
equivalence are shown in \cref{fig:llTraverseProduct,fig:llTraverseProductInv}.

Consider
the proof obligation $\{\phi_{{\tt S2:C2}}\} ({\tt S2\rightarrow S4\rightarrow S2}, {\tt C2\rightarrow C4\rightarrow C2}) \{sum_{S} = sum_{C}\}$, where
the node invariant $\phi_{{\tt S2:C2}}$ contains
the \recursiveRelation{} $l_{S}\indEq{}Clist^{{\tt lnode}}_{m}(l_{C})$.
The corresponding (simplified) first-order logic condition for this
proof obligation is:
($l_{S}\indEq{}Clist^{{\tt lnode}}_{m}(l_{C}) \land {\tt sum}_S={\tt sum}_C \land \neg({\tt l_S}\ is\ {\tt Nil})) \Rightarrow (({\tt sum}_S + l_S.val) = (sum_C + \structPointer{l}{m}{{\tt lnode}}{val}))$.
The problem here is that we do not know of an efficient
SMT encoding for the \recursiveRelation{} ($l_{S}\indEq{}Clist^{{\tt lnode}}_{m}(l_{C})$)
and we fail to construct a non-recursive proof obligation even
after $k$ unrollings of this \recursiveRelation{}.
Ignoring this \recursiveRelation{} will incorrectly (although soundly) evaluate
the proof obligation to false; however, for a successful equivalence
proof, we need
the proof discharge algorithm to evaluate it to true. Let's call this
requirement \circled{\small R1}.

Now, consider the proof obligation formed by correlating two iterations
of the loop in program $S$ with one iteration of the loop in program $C$,
$\{\phi_{{\tt S2:C2}}\} ({\tt S2\rightarrow S4\rightarrow S2\rightarrow S4\rightarrow S2}, {\tt C2\rightarrow C4\rightarrow C2}) \{sum_{S} = sum_{C}\}$.
The equivalent
first-order logic condition is:
($l_{S}\indEq{}Clist^{{\tt lnode}}_{m}(l_{C}) \land {\tt sum}_S={\tt sum}_C \land \neg({\tt l_S}\ is\ {\tt Nil}) \land \neg({\tt l_S.tail}\ is\ {\tt Nil})) \Rightarrow (({\tt sum}_S + l_S.val + l_S.tail.val) = (sum_C + \structPointer{l}{m}{{\tt lnode}}{val}))$.  After one unrolling
of $l_{S}\indEq{}Clist^{{\tt lnode}}_{m}(l_{C})$,
this proof obligation evaluates to false.
Whenever a proof
obligation evaluates to false, we
expect an ideal proof discharge algorithm to generate a
counterexample that falsifies the proof condition.
Let's call this
requirement \circled{\small R2}.
Recall that such counterexamples help in faster
convergence of our invariant inference and correlation algorithms.

To tackle requirements \circled{\small R1} and \circled{\small R2},
our proof discharge algorithm converts the
original proof obligation
$P: \{\phi_s\} (e) \{\phi_d\}$
into two approximated proof obligations:
$(P_{{\tt pre-o}}: \{\phi^{o_{d_1}}_s\} (e) \{\phi_d\})$
and
$(P_{{\tt pre-u}}: \{\phi^{u_{d_2}}_s\} (e) \{\phi_d\})$. Here
$\phi^{o_{d_1}}_s$ and
$\phi^{u_{d_2}}_s$ represent the over- and under-approximated
versions of precondition $\phi_s$ respectively, and $d_1$ and $d_2$ represent
{\em depth} parameters that indicate the degree of over- and
under-approximation.
To explain
our over- and under-approximation scheme, we
first need to
introduce the notion of the {\em depth of an ADT value}.

\subsubsection{Depth of an ADT value}
To define the depth of an ADT value,
we view the ADT as a context-free grammar:
\begin{itemize}
\item The set of terminals
consist of (a) scalar values, e.g., {\tt 42} for the {\tt i32} type, and
(b) the ADT value constructors (e.g., {\tt LNil}, {\tt LCons}).
\item The set of non-terminals are all the scalar type identifiers ({\tt i32}, {\tt bool}, {\tt unit})
and all the ADT type identifiers (e.g., {\tt List}, {\tt Tree}, {\tt Matrix2D}).
\item Productions of this context-free grammar specify the values that may be constructed for each type identifier: (a) For each basic type (e.g. {\tt i32}), we add productions from its type identifier to all values of that type, e.g., {\tt i32 $\rightarrow$ 0|1|2|...|($2^{32}-1$)};
(b) For an ADT, each constructor represents a separate production rule,
e.g., ADT {\tt List} has two productions: {\tt List} $\rightarrow$ {\tt LNil} | {\tt LCons} {\tt i32} {\tt List}.
% \begin{itemize}
% \item For each basic type (e.g. {\tt i32}), we add productions from its type identifier to all values of that type, e.g., {\tt i32 $\rightarrow$ 0|1|2|...|($2^{32}-1$)},
% \item For an ADT, each constructor represents a separate production rule,
% e.g. ADT {\tt List} has two productions: {\tt List} $\rightarrow$ {\tt LNil} | {\tt LCons} {\tt i32} {\tt List}.
% \end{itemize}
\item The start non-terminal is the top-level ADT identifier (e.g., {\tt List}).
\end{itemize}
In this context-free grammar interpretation of an ADT, a value of this ADT type
can be viewed as a {\em parse tree} (also called a derivation tree) of the
grammar.
The {\em depth} of a node in this parse tree is the number
of ADT identifiers (but not scalar type identifiers) in the
path from the root node to the node representing that terminal value (both inclusive).
\Cref{fig:parseTrees} shows examples of values of the {\tt List}, {\tt Tree},
and {\tt Matrix2D} ADTs, and the depth values of the parse tree nodes.
The {\em maximum depth of an ADT value} is the maximum depth of any node
in the parse tree of that value.

\begin{figure}[t]
\begin{tabular}{ccc}
\begin{subfigure}[b]{0.32\textwidth}
\begin{center}
{\includegraphics[scale=0.8]{chapters/figures/figParseTreeList3.pdf}}
\end{center}
\caption{\label{fig:listParseTree}{\small\tt List = LNil | \newline LCons(i32, List)}}
\end{subfigure}%
&
\begin{subfigure}[b]{0.22\textwidth}
\begin{center}
{\includegraphics[scale=0.78]{chapters/figures/figParseTreeTree2.pdf}}
\vspace{10px}
\end{center}
\caption{\label{fig:treeParseTree}{\footnotesize\tt Tree = TNil | \newline TCons(i32, Tree, Tree)}}
\end{subfigure}%
&
\begin{subfigure}[b]{0.4\textwidth}
\begin{center}
{\includegraphics[scale=0.78]{chapters/figures/figParseTreeMatrix2.pdf}}
\end{center}
\caption{\label{fig:matrixParseTree}{\tt Matrix = MNil | \newline MCons(List, Matrix)}}
\end{subfigure}%
\\
\end{tabular}
\vspace{-12px}
\caption{\label{fig:parseTrees}Parse Trees with Depths (shown as superscript) for {\tt List}, {\tt Tree}, and {\tt Matrix2D}}
\end{figure}

\vspace{-6px}
\subsubsection{Overapproximation and Underapproximation}
To overapproximate (underapproximate)
a precondition $\phi$, each
conjunctive \recursiveRelation{} in $\phi$ is overapproximated (underapproximated)
individually.

The $d$-depth overapproximated
version of a \recursiveRelation{}
$l_1\indEq{}l_2$ is written as $l_1\indEqDepth{d} l_2$, where
$\indEqDepth{d}$ represents the condition that the two ADT values
$l_1$ and $l_2$ are {\em recursively equal up to depth $d$}. i.e., all
{\em terminals} at depth $\leq d$ in the parse trees of both values
are identical; however, terminals at depths $>d$ can have
different values.
$l_1\indEqDepth{d}l_2$ (for finite $d$) is a weaker
condition than $l_1\indEq{}l_2$ (overapproximation);
$l_1\indEq{}l_2$
is equivalent to
$l_1\indEqDepth{\infty}l_2$.

The $d$-depth underapproximated
version of a \recursiveRelation{}
$l_1\indEq{}l_2$ is written
as $l_1\indEqUapprox{d}l_2$, where $\indEqUapprox{d}$ represents
the condition that the two ADT values $l_1$ and $l_2$ are
{\em recursively equal and bounded to depth $d$}, i.e.,
$l_1$, $l_2$ have a maximum
depth $\leq d$ {\em and} they are recursively equal up to depth $d$.
Thus, $l_1\indEqUapprox{d}l_2$ is equivalent
to
$((l_1\indEqDepth{d} l_2)\land\depthBound{d}{l_1}\land\depthBound{d}{l_2})$,
where $\depthBound{d}{l}$ represents the condition that the maximum
depth of $l$ is $d$.
$l_1\indEqUapprox{d}l_2$ (for finite $d$)
is a stronger condition than $l_1\indEq{}l_2$ (underapproximation)
as it ensures both equality and max-depth of both values.
$l_1\indEq{}l_2$
is equivalent to
$l_1\indEqUapprox{\infty}l_2$.

\subsubsection{SMT encoding of overapproximate and underapproximate proof obligations}
\label{sec:ouapprox}
Unlike the original \recursiveRelation{} $l_1\indEq{}l_2$,
$l_1\indEqDepth{d}l_2$ and
$l_1\indEqUapprox{d}l_2$ can be encoded using SMT through {\em unrolling
till depth $d$}:
\begin{itemize}
\item $l_1\indEqDepth{d}l_2$ is equivalent to the condition
that the parse tree structures
of the two values $l_1$ and $l_2$ (after $d$-unrolling)
are isomorphic till depth $d$ {\em and} the
corresponding values in both ($d$-depth)
isomorphic structures
are also equal.
$l_1\indEqDepth{d}l_2$ can be identified through d-unrolling followed by removal of all conjunctive clauses containing \recursiveRelations{}.

For example, for {\tt List} values $l_1$
and $l_2$, the
condition $l_1\indEqDepth{1}{\tt Clist}^{{\tt lnode}}_{m}({\tt p})$
can be one-unrolled to:
\begin{small}
\begin{center}
\underline{if} $l_1$ {\em is} {\tt LNil} \underline{then} {\tt LNil} \underline{else} {\tt LCons($l_1$.val,$l_1$.tail)}\\
$\indEqDepth{1}$\\
$\mathrm{\tt \underline{if}\ (p==0)\ \underline{then}\ }{\tt LNil}\mathrm{\tt \ \underline{else}\ {\tt LCons}(}\structPointer{p}{m}{{\tt lnode}}{val},\ {\tt Clist}^{{\tt lnode}}_m(\structPointer{p}{m}{{\tt lnode}}{next}))$.
\end{center}
\end{small}
The decomposition of the
condition shown above
reduces to the SMT-encodable condition:
\begin{small}
$$
((l_1\ \ is\ \ {\tt LNil}) \Leftrightarrow {\tt (p == 0)}) \land (\neg(l_1\ \ is\ \ {\tt LNil})\Rightarrow(l_1{\tt.val}=\structPointer{p}{m}{{\tt lnode}}{val}))
$$
\end{small}

%This condition relates the structure (by equating the \underline{if}-\underline{then}-\underline{else} path
%conditions) and values (by equating the terminals if their respective path conditions evaluate to true) up to
%depth one. It is silent on the structure and values of the parse tree nodes at depth $>1$.

%Unrolling can be performed by recursively substituting any ADT values
%with their \underline{if}-\underline{then}-\underline{else} decomposition.
%For example, to obtain a two-unrolled version of the condition
%$l_1\indEqDepth{1}{\tt Clist}^{{\tt lnode}}_{m}({\tt l}_C)$,
%we take all the ADT values in the one-unrolled version,
%namely {\tt $l_1$.tail} and ${\tt Clist}^{{\tt lnode}}_m\mathrm{(p}\xrightarrow[]{m}_{{\tt lnode}}\mathrm{\tt next})$,
%and substitute them with their respective
%\underline{if}-\underline{then}-\underline{else} decompositions.

\item $\depthBound{d}{l}$ is equivalent to the condition
that the parse-tree nodes at depths
$>d$ are unreachable. This is achieved by unrolling a \recursiveRelation{}
till depth $d$ and then asserting the unreachability
of
\underline{if}-\underline{then}-\underline{else}
paths that reach nodes with depth $>d$ (by checking the
satisfiability of their expression path conditions).
For example, for a {\tt List} value $l$,
the condition
$\depthBound{2}{l}$ is equivalent to
$(l\ \ is\ \ {\tt LNil}) \vee (\neg(l\ \ is\ \ {\tt LNil})\land(l{\tt .tail}\ \ is\ \ {\tt LNil}))$.
Similarly,
$\depthBound{2}{{\tt Clist}^{{\tt lnode}}_{m}({\tt p})}$
is equivalent to
$({\tt p}=0) \vee (\neg({\tt p}\neq 0)\land({\tt p}\xrightarrow[]{m}_{{\tt lnode}}\mathrm{\tt next}=0))$.
\end{itemize}

\subsubsection{Proof discharge algorithm for Category 2 proof obligations}
\label{sec:cat2algo}
Thus, for a {\em Category 2} proof obligation
$P: \{\phi_s\} (e) \{\phi_d\}$, we first
submit the proof obligation
$(P_{{\tt pre-o}}: \{\phi^{o_{d_1}}_s\} (e) \{\phi_d\})$
to the SMT solver. Recall that the precondition $\phi^{o_{d_1}}_s$
is the overapproximated version
of $\phi_s$.
If the SMT solver evaluates
$P_{{\tt pre-o}}$ to true, then we return true for
the original proof obligation $P$ --- if the
Hoare triple with an overapproximate precondition
holds, then the original Hoare triple
also holds.

If the
SMT solver evaluates
$P_{{\tt pre-o}}$ to false, then we submit
the proof obligation
$(P_{{\tt pre-u}}: \{\phi^{u_{d_2}}_s\} (e) \{\phi_d\})$
to the SMT solver. Recall that the precondition $\phi^{u_{d_2}}_s$
is the underapproximated version of $\phi_s$.
If the SMT solver evaluates
$P_{{\tt pre-u}}$ to false, then we return false for
the original proof obligation $P$ --- if the
Hoare triple with an underapproximate precondition
does not hold, then the original Hoare triple
also does not hold. Further, a counterexample that
falsifies $P_{{\tt pre-u}}$ would also falsify $P$,
and is thus usable in invariant inference and correlation procedures.
% and is thus a valid counterexample for use in invariant
% inference and correlation procedures.

Finally, if the SMT solver evaluates
$P_{{\tt pre-u}}$ to true, then we have neither
proven nor disproven $P$. In this case, we
imprecisely (but soundly) return false for the
original proof obligation $P$ (without a counterexample).
Revisiting our examples,
the proof obligation $\{\phi_{{\tt S2:C2}}\} ({\tt S2\rightarrow S4\rightarrow S2}, {\tt C2\rightarrow C4\rightarrow C2}) \\ \{sum_{S} = sum_{C}\}$ is provable using a depth-1 overapproximation of the
precondition $\phi_{{\tt S2:C2}}$ --- the depth-1 overapproximation retains the
information that the first value in lists ${\tt l}_S$
and ${\tt l}_C$ are equal, and that is sufficient to prove that
the new values of $sum_{S}$ and $sum_{C}$ are also equal (given that the
old values are equal, as encoded in $\phi_{{\tt S2:C2}}$).

Similarly, the proof obligation
$\{\phi_{{\tt S2:C2}}\} ({\tt S2\rightarrow S4\rightarrow S2\rightarrow S4\rightarrow S2}, {\tt C2\rightarrow C4\rightarrow C2}) \{sum_{S} = sum_{C}\}$ evaluates to false (with a counterexample) using
a depth-2 underapproximation of the precondition $\phi_{{\tt S2:C2}}$.
In the depth-2 underapproximate version, we try to prove that
if the equal lists $l_S$ and ${\tt Clist}^{{\tt lnode}}_{m}(l_C)$
have exactly two
nodes\footnote{The underapproximation
restricts both lists to have at most
two nodes; the path condition for ${\tt S2\rightarrow S4\rightarrow S2\rightarrow S4\rightarrow S2}$ additionally
restricts $l_S$ to have at least two nodes; together, this is equivalent to the list having
exactly two nodes}, then
the sum of the values in the two nodes of $l_S$ is equal to the
value stored in the first node in {\tt l}$_C$.
This proof obligation will return a counterexample that
maps program variables to their concrete values. We show a
possible counterexample to this proof obligation below.
%
\begin{center}
\begin{footnotesize}
\begin{tabular}{cc|c}
{\tt sum$_S$ $\mapsto$ 3} & {\tt sum$_C$ $\mapsto$ 3} & \multirow{3}{*}{
$m$ $\mapsto$ $\Bigg($
\begin{tabular}{l}
{\tt 0x123} $\mapsto_{\tt lnode}$ (.value $\mapsto$ 42, .next $\mapsto$ {\tt 0x456}), \\
{\tt 0x456} $\mapsto_{\tt lnode}$ (.value $\mapsto$ 43, .next $\mapsto$ {\tt 0x000}), \\
() $\mapsto$ {\tt 77} \\
\end{tabular}
$\Bigg)$
} \\
\multicolumn{2}{l|}{{\tt l$_S$} $\mapsto$ {\tt LCons(42,LCons(43,LNil))}} \\
\multicolumn{2}{l|}{{\tt l$_C$} $\mapsto$ {\tt 0x123}} \\
\end{tabular}
\end{footnotesize}
\end{center}
% \begin{center}
% \begin{footnotesize}
% \begin{tabular}{l@{ $\mapsto$ }l|l}
% {\tt sum$_S$} & {\tt 3} &
% \multirow{4}{*}{
% $m$ $\mapsto$ $\Bigg($
% \begin{tabular}{l}
% {\tt 0x123} $\mapsto_{\tt lnode}$ (.value $\mapsto$ 42, .next $\mapsto$ {\tt 0x456}), \\
% {\tt 0x456} $\mapsto_{\tt lnode}$ (.value $\mapsto$ 43, .next $\mapsto$ {\tt 0x000}), \\
% () $\mapsto$ {\tt 77} \\
% \end{tabular}
% $\Bigg)$
% }
% \\
% {\tt sum$_C$} & {\tt 3} \\
% ${\tt l}_S$   & {\tt LCons(42,LCons(43,LNil))} \\
% ${\tt l}_C$   & {\tt 0x123} \\
% \end{tabular}
% \end{footnotesize}
% \end{center}

% \begin{center}
% \begin{tabular}{ll@{ $\mapsto$ }l}
% \{ & {\tt sum$_S$} & {\tt 3},\\
%    & {\tt sum$_C$} & {\tt 3},\\
%    & ${\tt l}_S$ & {\tt LCons(42,LCons(43,LNil))},\\
%    & ${\tt l}_C$ & {\tt 0x123},\\
%    & $m$ & $\bigg($
%            \begin{tabular}{ll}
%              & {\tt 0x123} $\mapsto_{\tt lnode}$ (.value $\mapsto$ 42, .next $\mapsto$ {\tt 0x456}),\\
%               & {\tt 0x456} $\mapsto_{\tt lnode}$ (.value $\mapsto$ 43, .next $\mapsto$ 0),\\
%               & () $\mapsto$ {\tt 77}\\
%            \end{tabular}$\bigg)$ \\
% \}\\
% \end{tabular}
% \end{center}

This counterexample maps variables to values (e.g., {\tt sum}$_C$ maps to an {\tt i32} value {\tt 3}
and {\tt l}$_S$ maps to a {\tt List} value {\tt LCons(42,LCons(43,LNil))}).
It also maps the C program's memory state $m$ to an array that
maps the regions starting at addresses {\tt 0x123} and {\tt 0x456} (regions of size
`{\tt sizeof lnode}') to memory objects of type {\tt lnode} (with the
{\tt value} and {\tt next} fields shown for each object). For all other addresses (except
the ones for which an explicit mapping is available), $m$ maps them
to the default byte-value {\tt 77} (shown
as {\tt () $\mapsto$ 77}) in this counterexample.

This counterexample satisfies the preconditions $l_S\indEqUapprox{2}{\tt Clist}^{{\tt lnode}}_{m}({\tt l}_C)$
and $sum_S=sum_C$. Further, when the
paths $({\tt S2\rightarrow S4\rightarrow S2\rightarrow S4\rightarrow S2}, {\tt C2\rightarrow C4\rightarrow C2})$
are executed starting at the machine state represented by this counterexample, the resulting
values of $sum_S$ and $sum_C$ are {\tt 3+42+43=88} and {\tt 3+42=45} respectively. Evidently, the
counterexample falsifies the proof condition because these values are not equal (as required by the postcondition).
\vspace{-6px}
\subsection{Category 3: The $k$-unrolled form of $P$ contains a \recursiveRelation{} in RHS and optionally LHS as well}
\label{sec:cat3}
In \cref{fig:llAllocProductCFG}, consider a proof obligation generated
across the product-CFG edge ${\tt (S3:C5) \rightarrow (S3:C3)}$
while checking if the {\small $\circled{I4}$} invariant, $l_{S}\indEq{}Clist^{lnode}_{m}(l_{C})$,
holds at {\tt (S3:C3)}:
$\{\phi_{{\tt S3:C5}}\} ({\tt S3\rightarrow S5\rightarrow S3}, \\ {\tt C5\rightarrow C3}) \{l_{S}\indEq{}Clist^{lnode}_{m}(l_{C})\}$.
Here, a \recursiveRelation{} is present both in the precondition $\phi_{{\tt S3:C5}}$ ({\small \circled{I8}})
and in the postcondition ({\small \circled{I4}}) and we could
not remove them even after $k$ unrollings and unification.
When lowered to first-order logic
through {\tt WP$_{{\tt S3\rightarrow S5\rightarrow S3},{\tt C5\rightarrow C3}}$}, this translates to (showing only relevant
relations):
\vspace{-5px}
\begin{small}
\begin{equation}\label{eqn:ex1cat3}
({\tt i}_S={\tt i}_C \land {\tt p}_C={\tt malloc()} \land {\tt l}_S \indEq{} {\tt Clist}_m^{{\tt lnode}}({\tt l}_C)) \Rightarrow ({\tt LCons}({\tt i}_S, {\tt l}_S) \indEq{} {\tt Clist}_{m'}^{{\tt lnode}}({\tt p}_C))
\end{equation}
\end{small}
On the {\tt RHS} of this first-order logic formula, {\tt LCons(i, l$_S$)} is compared for
equality with {\tt Clist$_{m'}^{{\tt lnode}}$(p$_C$)}; here {\tt p$_C$}
represents the address of the newly allocated {\tt lnode} object (through {\tt malloc}) and $m'$
represents the C memory state after executing the writes at lines
{\tt C5} and {\tt C6} on the path {\tt C5$\rightarrow$C3},
i.e.,
\vspace{-5px}
\begin{small}
\begin{equation}\label{eqn:memstore}
m' \equiv m[{\tt \&(p}_C\xrightarrow[]{m}_{\mathrm{\tt lnode}}{\tt value}) \leftarrow {\tt i}_C]_{\tt i32}[{\tt \&(p}_C\xrightarrow[]{m}_{\mathrm{\tt lnode}}{\tt next}) \leftarrow {\tt l}_C]_{\tt i32}
\end{equation}
\end{small}
Here, $m[{\tt a}\leftarrow{\tt v}]$ reprsents an array that is
formed by writing value {\tt v} to address {\tt a} in array $m$.
We also refer to
these memory writes that distinguish $m$ from $m'$, the {\em distinguishing writes}.

\subsubsection{Replacing \SpecL{} recursive values in the RHS with lifted C values}
\toolName{} utilizes
the $\indEq{}$ relationships in the {\tt LHS} (antecedant) of ``$\Rightarrow$''
to rewrite \cref{eqn:ex1cat3}
so that the recursive {\tt List} values in its {\tt RHS} (conclusion)
are replaced with the lifted $C$ values (lifted using
the {\tt Clist} constructor). Thus, we
rewrite \cref{eqn:ex1cat3} to:
%\begin{equation}\label{eqn:clistsEqualUnder}
%({\tt l}_S \indEq{} {\tt Clist}_m^{{\tt lnode}}({\tt l}_C)) \Rightarrow
%({\tt Clist}_m^{{\tt lnode}}({\tt l}_C) \indEq{} {\tt Clist}_{m'}^{{\tt lnode}}({\tt l}_C))
%\end{equation}
\vspace{-5px}
\begin{small}
\begin{equation}\label{eqn:ex2cat3}
({\tt i}_S={\tt i}_C \land {\tt p}_C={\tt malloc()} \land {\tt l}_S \indEq{} {\tt Clist}_m^{{\tt lnode}}({\tt l}_C)) \Rightarrow ({\tt LCons}({\tt i}_S, {\tt Clist}_m^{{\tt lnode}}({\tt l}_C)) \indEq{} {\tt Clist}_{m'}^{{\tt lnode}}({\tt p}_C))
\end{equation}
\end{small}
After decomposition and RHS-breaking, \cref{eqn:ex1cat3}
reduces to the following smaller proof obligations (showing only the
{\tt RHS}, the {\tt LHS} is the same as in \cref{eqn:ex1cat3}):
(1) $\neg({\tt p}_C=0)$,
(2) {\tt $\neg({\tt p}_C=0)$ $\rightarrow$ i$_S$\ =\ (p$_C\xrightarrow[]{m'}_{\mathrm{\tt lnode}}$value)}, and
(3) {\tt $\neg({\tt p}_C=0)$ $\rightarrow$ Clist$_{m}^{{\tt lnode}}$(l$_C$)$\indEq{}$Clist$_{m'}^{{\tt lnode}}$(p$_C\xrightarrow[]{m'}_{\mathrm{\tt lnode}}$next)}.
The first two proof obligations fall in {\em Category 2} and
are discharged through over- and under-approximation (as discussed
in \cref{sec:cat2}):
\begin{enumerate}
\item The first proof obligation with
postcondition $\neg({\tt p}_C=0)$
evaluates to true because the {\tt LHS} ensures that {\tt p$_C$} is the
return value of an allocation function ({\tt malloc}) which must be
non-null due to the $(C\ \  \mathrm{\tt fits})$ assumption.
%Henceforth, we omit $\neg({\tt p}_C=0)$ from the antecedant of postconditions (2) and (3)
%since in either case they evaluate to true because of $(C\ \  \mathrm{\tt fits})$ assumption.
\item The second proof obligation with
postcondition {\tt (i$_S$=(p$_C\xrightarrow[]{m'}_{\mathrm{\tt lnode}}$value))}
also evaluates to true because {\tt i$_C$} is written to address {\tt \&(p$_C\xrightarrow[]{m'}_{\mathrm{\tt lnode}}$value)}
in $m'$ (\cref{eqn:memstore}) and the {\tt LHS} ensures that {\tt i$_S$=i$_C$}.
\end{enumerate}

For ease of exposition, we simplify the postcondition of the third proof obligation
from \\
{\small \tt $\neg({\tt p}_C=0)$ $\rightarrow$ (Clist$_{m}^{{\tt lnode}}$(l$_C$)$\indEq{}$Clist$_{m'}^{{\tt lnode}}$(p$_C\xrightarrow[]{m'}_{\mathrm{\tt lnode}}$next))}
to
{\tt (Clist$_{m}^{{\tt lnode}}$(l$_C$)$\indEq{}$Clist$_{m'}^{{\tt lnode}}$(l$_C$))}.
This simplification is valid because {\tt l$_C$}
is written
to address {\tt \&(p$_C\xrightarrow[]{m'}_{\mathrm{\tt lnode}}$next)}
in $m'$ (\cref{eqn:memstore}).
Also, we have already
shown that $\neg({\tt p}_C=0)$ holds.
Thus, the third proof obligation can be rewritten as
a \recursiveRelation{} between two lifted expressions\footnote{This
simplification-based rewriting is only
shown for ease of exposition,
and has no effect on the operation
of the algorithm.
Even if the proof obligation is not simplified,
the unification-based proof discharge algorithm will generate
proof conditions of the form $\neg({\tt p}_C=0)\Rightarrow{}$((p$_C\xrightarrow[]{m'}_{\mathrm{\tt lnode}}$next)=l$_C$)
which will be successfully discharged by the SMT solver.}:
\begin{small}
\begin{equation}\label{eqn:clistPreserved}
{\tt Clist}_m^{\tt lnode}({\tt l}_C) \indEq{} {\tt Clist}_{m'}^{\tt lnode}({{\tt l}_C})
\end{equation}
\end{small}

Thus, we are interested in proving a \recursiveRelation{}
between two {\tt List} values in $C$ under different memory states $m$ and $m'$.
%Note that in general this \recursiveRelation{} may be under an precondition.
We next show
how such a \recursiveRelation{} can be converted to a bisimulation proof.

% Thus,
% under an antecedant that ${\tt Clist}_m^{{\tt lnode}}({\tt l}_C)$
% is recursively equal to a {\tt List} value {\tt l}$_S$ in \SpecL{},
% we are interested in proving
% ${\tt Clist}_m^{{\tt lnode}}({\tt l}_C) \indEq{}
% {\tt Clist}_{m'}^{{\tt lnode}}({\tt l}_C)$.  We next show how this
% \recursiveRelation{} can be converted to a bisimulation proof.

%Informally, this proof obligation in \cref{eqn:clistPreserved} checks
%that the structure
%and values of the linked list pointed-to by {\tt l}$_C$
%in $m$ remain preserved even after the
%memory writes to the newly allocated node (in $m'$).
%At a minimum, such reasoning requires an alias analysis
%that confirms that the newly allocated node is isolated from
%the existing linked list nodes.  We discuss this in more
%detail in the following section.

\subsubsection{Converting recursive equality between lifted expressions to
a bisimulation}
\label{sec:recursiveEqToBisim}
Consider a
program that recursively applies
the unrolling procedure in \cref{eqn:clist}
to deconstruct
${\tt Clist}_m^{{\tt lnode}}({\tt l}_C)$.
For example,
${\tt Clist}_m^{{\tt lnode}}({\tt l}_C)$
may yield a recursive call
to the unrolling procedure
${\tt Clist}_m^{{\tt lnode}}(\structPointer{{\tt l}_C}{m}{\tt lnode}{next})$
and so on, until the ${\tt l}_C$ argument to
the unrolling procedure becomes null.
This program essentially deconstructs ${\tt Clist}_m^{{\tt lnode}}({\tt l}_C)$
into its {\em scalar} values and reconstructs
a {\tt List} value equal to the value
represented by ${\tt Clist}_m^{{\tt lnode}}({\tt l}_C)$.
We call this program a {\em reconstruction program} based
on the unrolling procedure of
${\tt Clist}_m^{{\tt lnode}}({\tt l}_C)$.
% We call this program that deconstructs
% ${\tt Clist}_m^{{\tt lnode}}({\tt l}_C)$
% into its components and reconstructs an output
% {\tt List} value,
% a {\em reconstruction program} based on the unrolling procedure of
% ${\tt Clist}_m^{{\tt lnode}}({\tt l}_C)$.
%(using the unrolling procedure in \cref{eqn:clist}) until
%a null pointer is reached
%yields a sequence of
%${\tt Clist}_m^{{\tt lnode}}({\tt l}^0_C)$,
%${\tt Clist}_m^{{\tt lnode}}({\tt l}^1_C)$,
%${\tt Clist}_m^{{\tt lnode}}({\tt l}^2_C)$, ...
%values, where each
%{\tt l}$^i_C$ (for $i$=0,1,2,...)
%represents a C pointer in $m$ that is lifted
%to a {\tt List} value (using ${\tt Clist}_m^{{\tt lnode}}$)
%during this deconstruction. If this sequence is finite,
%its last value
%is {\tt LNil}.
%We call this series of C pointers
%${\tt l}^0_C$,
%${\tt l}^1_C$,
%${\tt l}^2_C$, ..., the {\tt Clist}$_m^{{\tt lnode}}$-reachable
%pointers of {\tt l}$^0_C$, or \Reachable{{\tt Clist}_m^{{\tt lnode}}}{${\tt l}^0_C$}.

%\begin{definition}
%A lifted ${\tt Clist}_m^{{\tt lnode}}({\tt l}_C))$ value
%is {\em well-formed} iff there exists a {\tt List} value ${\tt l}_S$
%in the \SpecL{} program $S$ such that
%$({\tt l}_S \indEq{} {\tt Clist}_m^{{\tt lnode}}({\tt l}_C))$.
%\end{definition}
%
%\begin{lemma}\label{lemma:finitenessOfWellFormed}
%For a well-formed lifted
%value ${\tt Clist}_m^{{\tt lnode}}({\tt l}_C))$,
%\Reachable{{\tt Clist}_m^{{\tt lnode}}}{${\tt l}^0_C$} is finite.
%\end{lemma}
%\begin{proof}
%\Cref{lemma:finitenessOfWellFormed} follows from the definition of well-formedness and the finiteness
%of any ADT value in the \SpecL{} language.
%\end{proof}
%
%\begin{theorem}\label{theorem:clistsEqual}
%For a well-formed lifted value ${\tt Clist}_m^{{\tt lnode}}({\tt l}_C))$:
%\begin{gather*}
%{\tt Clist}_m^{{\tt lnode}}({\tt l}_C)
%\indEq{}
%{\tt Clist}_{m'}^{{\tt lnode}}({\tt l}_C)\\
%\Leftrightarrow\\
%(\ReachableMath{{\tt Clist}_m^{{\tt lnode}}}{{\tt l}_C} = \ReachableMath{{\tt Clist}_{m'}^{{\tt lnode}}}{{\tt l}_C}\ \ \land\\
%\forall_{{{\tt l}^i_C\in\ReachableMath{{\tt Clist}_m^{{\tt lnode}}}{{\tt l}_C}}}({\structPointer{l^i_C}{m}{s}{value}}={\structPointer{l^i_C}{m'}{s}{value}} \land\ {\structPointer{l^i_C}{m}{s}{next}}={\structPointer{l^i_C}{m'}{s}{next}}))
%\end{gather*}
%\end{theorem}
%\begin{proof}
%The proof proceeds by induction on the structure of the
%well-formed value ${\tt Clist}_m^{{\tt lnode}}({\tt l}_C))$.
%The `{\tt l}$_C=0$' condition represents the base case.
%The induction step involves unifying the expansions of
%${\tt Clist}_m^{{\tt lnode}}$
%and
%${\tt Clist}_{m'}^{{\tt lnode}}$ obtained using \cref{eqn:clist}, and using the induction hypothesis.
%\end{proof}

\begin{theorem}\label{theorem:clistsEqual}
Under the antecedant
$({\tt l}_S \indEq{} {\tt Clist}_m^{{\tt lnode}}({\tt l}_C))$:

$({\tt Clist}_m^{{\tt lnode}}({\tt l}_C)
\indEq{}
{\tt Clist}_{m'}^{{\tt lnode}}({\tt l}_C))$
holds iff a bisimulation relation
exists between the two programs that
deconstruct
${\tt Clist}_m^{{\tt lnode}}({\tt l}_C)$
and
${\tt Clist}_{m'}^{{\tt lnode}}({\tt l}_C)$.
The bisimulation relation must ensure that the
observables ({\tt List} values) generated by both procedures are
identical.
\end{theorem}
\begin{proof}\let\qed\relax
The ``if'' case of this ``iff'' relation follows
from noting that the observable values of
a reconstruction program are the
generated {\tt List} values. Thus, a
successful bisimulation
check ensures equal
{\tt List}
values upon termination. Termination
follows from the antecedant because the
\SpecL{} value $l_S$ must be finite.

The ``only if'' case
follows from the unification of the
unrolling procedure (in \cref{eqn:clist}) for
${\tt Clist}_m^{{\tt lnode}}({\tt l}_C)$
and
${\tt Clist}_{m'}^{{\tt lnode}}({\tt l}_C)$.

%The proof proceeds by induction on the structure of the
%${\tt Clist}_m^{{\tt lnode}}({\tt l}_C))$.
%The base case is
%represented by the `{\tt l}$_C=0$' condition ---
%both deconstruction procedures terminate in this case (irrespective
%of the contents of $m$ and $m'$).
%The induction step involves unifying the expansions of
%${\tt Clist}_m^{{\tt lnode}}$
%and
%${\tt Clist}_{m'}^{{\tt lnode}}$ obtained using \cref{eqn:clist}, and using the induction hypothesis.
%At the induction step, we need to show that the first procedure takes the \underline{if} branch
%iff the second procedure takes the \underline{if} branch.  Further, we need
%to show that whenever the {\tt value}
%and {\tt next} fields are read
%in the {\tt lnode} objects pointed-to by {\tt l$_C$} in $m$ and $m'$
%respectively, they have identical contents.
\end{proof}

Thus, to check if ${\tt Clist}_m^{{\tt lnode}}({\tt l}_C)
\indEq{} {\tt Clist}_{m'}^{{\tt lnode}}({\tt l}_C)$, we
check if a bisimulation exists
between the two respective reconstruction programs (potentially
under an antecedant).  \Cref{theorem:clistsEqual} generalizes
to an arbitrary ADT and to an arbitrary proof obligation that
relates two lifted values constructed from potentially different
$C$ values and memory states.

\subsubsection{Checking bisimulation between the reconstructions of two ADT values}
\label{sec:bisim}
To check bisimulation, we attempt to show that both reconstructions
proceed in lockstep, and the invariants at
each step of this lockstep execution ensure equal observables.
We use a product-CFG to encode this lockstep execution --- to distinguish this
product-CFG from the top-level product-CFG that relates $S$ and $C$, we call
this product-CFG that relates two reconstruction programs, a {\em reconstruction product-CFG}
or {\em recons-PCFG} for short.
% i.e., (1) the first deconstruction procedure evaluates
%an \underline{if} condition (e.g., {\tt l$_C$=0}) to true iff the second
%deconstruction procedure also evaluates the
%corresponding \underline{if} condition to true; and (2)
%the values observed by the first deconstruction procedure (e.g., {\tt lnode.value}
%which is used for constructing the observable {\tt List}) are identical to the
%ones observed by the second deconstruction procedure.
%As we have seen earlier, this lockstep execution can be encoded
%as a product-CFG: the invariants at the nodes of this product-CFG
%represent the conditions that must hold at the correlated
%PCs of this
%lockstep execution; for
%a successful bisimulation check, these invariants must ensure equal observables.


\begin{figure}
\begin{tabular}{cc}
\begin{subfigure}[b]{0.50\textwidth}
\includegraphics[scale=0.75]{figClistCfg.pdf}
\caption{\label{fig:deconsProg}Deconstruction Program}
\end{subfigure}%
&
\begin{subfigure}[b]{0.50\textwidth}
\includegraphics[scale=0.75]{figClistProductCfg.pdf}
\caption{\label{fig:deconsPCFG}Decons-PCFG}
\end{subfigure}%
%&
%\begin{subfigure}[b]{0.17\textwidth}
%\includegraphics[scale=0.8]{figMallocPointsToGraph.pdf}
%\caption{\label{xxx}XXX}
%\end{subfigure}%
\\
\end{tabular}
\vspace{-8px}
\caption{\label{fig:decons}The deconstruction program and the decons-PCFG for {\tt Clist$^{lnode}_m({\tt l_{C}})$}. In \cref{fig:deconsProg}, {\tt D0} represents the unrolling procedure entry node, and the square boxes show the transfer functions of the unrolling procedure (\cref{eqn:clist}). The dashed edges represent a recursive function call. In \cref{fig:deconsPCFG}, the square box to the right of node {\tt D0:D0} contains the inferred invariants for this decons-PCFG.}
\vspace{-8px}
\end{figure}


The reconstruction
program and the recons-PCFG for our {\tt Clist} example
are shown in \cref{fig:decons}.
To check bisimulation
between the programs that deconstruct
${\tt Clist}_m^{{\tt lnode}}({\tt l}_C)$
and ${\tt Clist}_{m'}^{{\tt lnode}}({\tt l}_C)$, the recons-PCFG
correlates one unrolling of the first
program
with one
unrolling of the second program.
An unrolling of each reconstruction
program is based on the unrolling procedure in \cref{eqn:clist}.
Thus, the PC-transition correlations of both programs
are trivially obtained
by unifying the unrolling procedure with itself. A node
is created in the recons-PCFG that
encodes the correlation of the entries of the unrolling procedure in
both programs, we call this node the {\em recursive-node} in the
recons-PCFG, e.g., the
recursive node in \cref{fig:reconsPCFG} is {\tt D0:D0}. A recursive
call becomes a back-edge in the recons-PCFG that terminates at the
recursive-node.
A candidate
invariant at the recursive-node
is obtained by equating the pair of corresponding
{\tt l}$_C$ variables across the first and second
programs, i.e.,
${\tt l}^{fst}_C={\tt l}^{snd}_C$.
At the start of
both reconstruction
programs, ${\tt l}^{fst}_C={\tt l}^{snd}_C={\tt l}^{start}_C$
--- the
same ${\tt l}^{start}_C$ is passed to both reconstruction
programs, only the memory states $m$ and $m'$ are different.
The bisimulation check thus involves checking that
if the invariant
${\tt l}^{fst}_C={\tt l}^{snd}_C$
holds at the recursive-node,
then during one iteration of the unrolling procedure
in both programs:
\begin{enumerate}
\item The \underline{if} condition
({\tt l$^{fst}_C=0$}) in the first program
is equal to the corresponding \underline{if}
condition ({\tt l$^{snd}_C=0$}) in the
second program.
\item If the \underline{if} condition
evaluates to false in both programs, then
the observable values (that are used in the
construction of the list) are equal, i.e.,
$(({\tt l}^{fst}_C\neq 0)\land({\tt l}^{snd}_C\neq 0))\Rightarrow (\structPointer{{\tt l}^{fst}_C}{m}{s}{val}=
\structPointer{{\tt l}^{snd}_C}{m'}{\tt lnode}{val})$.
\item If the \underline{if} condition
evaluates to false in both programs, then
the invariant holds at the beginning
of the unrolling procedure invoked through the
recursive call.
This involves checking equality
of the arguments to the recursive call, i.e.,
$(({\tt l}^{fst}_C\neq 0)\land({\tt l}^{snd}_C\neq 0))\Rightarrow \structPointer{{\tt l}^{fst}_C}{m}{\tt lnode}{next}
=
\structPointer{{\tt l}^{snd}_C}{m'}{\tt lnode}{next}$.
\end{enumerate}
The first check succeeds due to the invariant
${\tt l}^{fst}_C={\tt l}^{snd}_C$.
For the second and third checks, we additionally
need to reason that the memory objects
$\structPointer{{\tt l}^{fst}_C}{m}{\tt lnode}{val}$ and
$\structPointer{{\tt l}^{fst}_C}{m}{\tt lnode}{next}$ cannot
alias with the writes (in $m'$ in \cref{eqn:memstore})
to the newly allocated objects
${\tt p}_C\xrightarrow[]{m}_{\mathrm{\tt lnode}}{\tt value}$
and
${\tt p}_C\xrightarrow[]{m}_{\mathrm{\tt lnode}}{\tt next}$.
This aliasing information is captured using a points-to analysis,
described in \cref{sec:pointsTo}.

Notice that a bisimulation check
between the reconstruction programs is significantly
easier than the top-level bisimulation check between \SpecL{}
and C programs: here,
the correlation of PC transitions is trivially
identified by unifying the unrolling procedure with itself, and
the candidate invariants
are obtained by equating each corresponding pair
of variables across
the two programs.

\subsubsection{Points-to Analysis}
\label{sec:pointsTo}
To reason about aliasing (as required during the bisimulation
check in \cref{sec:bisim}), we conservatively compute the
{\em may-point-to}
information for each program value using Andersen's algorithm \cite{andersen94programanalysis}.
The range of this computed
may-point-to function are {\em sets of region labels}, where
each region label identifies a set of memory objects.
The sets of memory objects identified by two distinct region
labels are necessarily disjoint. We write $p\pointsTo\{R_1,R_2\}$
to represent the condition that value $p$ may point to
an object belonging to one of the region labels $R_1$ or $R_2$ (but
may not point to any object outside of $R_1$ and $R_2$).

We populate the set of all region labels using the
{\em allocation
sites} of the program, i.e., PCs where a call to
{\tt malloc} exists, e.g., {\tt C4}
in \cref{fig:llAllocCIR} is an allocation site.
For each allocation site $A$, we create
two region labels: (1) the first region label, called $A_1$,
identifies the set of memory
objects that were allocated by the most recent execution of $A$. (2) The second
region label, called $A_{2+}$, identifies
the set of memory objects that were allocated by older (not the most
recent) executions of $A$.

For example, at the start of PC {\tt C7} in \cref{fig:llAllocCIR},
{\tt i}$_C\pointsTo{}\{\}$,
{\tt n}$_C\pointsTo{}\{{\tt C4}_1\}$,
and {\tt l}$_C\pointsTo{}\{{\tt C4}_{2+}\}$.
Because the may-point-to analysis determines the
sets of objects pointed-to by {\tt n}$_C$ and {\tt l}$_C$ to
be disjoint ($\{{\tt C4_{1}}\}$ vs. $\{{\tt C4_{2+}}\}$), any
memory accesses through {\tt n}$_C$ and {\tt l}$_C$
cannot alias at {\tt C7} (for an access
offset that is within the bounds of
the allocation size `{\tt sizeof lnode}').

The may-point-to information is computed not
just for scalar program values ({\tt n}$_C$,
{\tt l}$_C$, ...) but also for each region label.
For region labels $A1_{r1}$, $A2_{r2}$, $A3_{r3}$:
$A1_{r1}\pointsTo{}\{A2_{r2},A3_{r3}\}$ represents
the condition that the values (pointers) stored
in objects identified
by $A1_{r1}$ may point to an object identified by
either $A2_{r2}$ or $A3_{r3}$ (but not to any object
outside $A2_{r2}$ and $A3_{r3}$).
In \cref{fig:llAllocCIR}, at PC {\tt C7}, we
get
${\tt C4}_{1}\pointsTo{}\{{\tt C4}_{2+}\}$ and
${\tt C4}_{2+}\pointsTo{}\{{\tt C4}_{2+}\}$.
The condition
${\tt C4}_{1}\pointsTo{}\{{\tt C4}_{2+}\}$
holds because the {\tt next} pointer of the object
pointed-to by ${\tt n}_C$ (which
is a ${\tt C4}_1$ object) may point to
a {\tt C4}$_{2+}$ object (e.g., object pointed-to
by {\tt l}$_C$).
Similarly, ${\tt C4}_{2+}\pointsTo{}\{{\tt C4}_{2+}\}$
says that a pointer within a {\tt C4}$_{2+}$ object
may point to a {\tt C4}$_{2+}$
object (but not to a {\tt C4}$_1$ object).

\subsubsection{Transferring points-to information to the recons-PCFG}
\label{sec:pointsToAsInvariants}
Recall that
in \cref{sec:recursiveEqToBisim},
we reduce a validity check of the condition
${\tt Clist}_m^{{\tt lnode}}({\tt l}_C)
\indEq{} {\tt Clist}_{m'}^{{\tt lnode}}({\tt l}_C)$
to a bisimulation check. Also, recall
that we discharge the bisimulation check through the construction
of a recons-PCFG that compares the unrolling procedure with itself (executing
on memory states $m$ and $m'$).
During this bisimulation check, we need to prove
that for each execution
of the unrolling procedure, $\structPointer{{\tt l}_C}{m}{\tt lnode}{\{val,next\}}$
and $\structPointer{{\tt l}_C}{m'}{\tt lnode}{\{val,next\}}$\footnote{Here, we use the symbol {\tt l}$_C$ to
refer to equal values {\tt l}$^{fst}_C$ and
{\tt l}$^{snd}_C$.} are equal.
To successfully discharge these proof obligations, it suffices
to show ${\tt l}_C$ cannot alias with the memory writes that
distinguish $m$ from $m'$.

Our points-to analysis on
the C program determines that at PC {\tt C5} (the start of
the product-CFG edge {\tt (S3:C5)$\rightarrow$(S3:C3)} across which the proof
condition is being evaluated),
the pointer to the {\em head}
of the list, i.e., ${\tt l}^{start}_C$
points to {\tt C4}$_{2+}$. It also determines that the distinguishing
writes modify memory regions belonging to {\tt C4}$_1$. 
Further, we get ${\tt C4}_{2+}\pointsTo{}\{{\tt C4}_{2+}\}$ at PC {\tt C5}.
However,
notice that these determinations only rule out aliasing of the list-head with
the distinguishing writes. We also need to confirm non-aliasing
of the internal nodes of the linked list with the distinguishing
writes.
For this, we need to identify a points-to invariant,
{\tt l}$_C\pointsTo{}\{{\tt C4}_{2+}\}$, at the recursive-node
of the recons-PCFG
(shown in \cref{fig:reconsPCFG}).
To see why {\tt l}$_C\pointsTo{}\{{\tt C4}_{2+}\}$ is
an inductive invariant at the recursive-node:
\begin{itemize}
\item (Base case) The invariant holds
at entry to the recons-PCFG (because it holds for ${\tt l}^{start}_C$).
\item (Induction step) If ${\tt l}_C\pointsTo{}\{{\tt C4}_{2+}\}$
holds at the start of an unrolling procedure,
it also holds at the start of a recursive call to the
unrolling procedure. This
follows from ${\tt C4}_{2+}\pointsTo{}\{{\tt C4}_{2+}\}$ (points-to information at PC {\tt C5}),
which ensures that {\tt l$_C$->next} may point to only ${\tt C4}_{2+}$ objects.
\end{itemize}

To identify this points-to invariant, we run
our points-to analysis (the same analysis that is run
on the C program) on the reconstruction programs (\cref{fig:deconsProg})
before
comparing them for equivalence. The boundary
condition for the points-to analysis at the
entry node of the reconstruction
program (e.g., {\tt D0} in \cref{fig:decons}) is based on
the results of the points-to analysis
on $C$ at the PC where the proof obligation is being discharged (e.g., {\tt C5}
in our \cref{fig:llAllocC}). The points-to invariants
at a node {\tt (l$^{fst}_i$,l$^{snd}_j$)} of a recons-PCFG are
derived from the results of the points-to analysis on the individual
reconstruction programs at nodes l$^{fst}_i$ and l$^{snd}_j$
respectively.

During proof obligation discharge (e.g., during the bisimulation
check on recons-PCFG), the points-to invariants are encoded as SMT constraints.
This allows us to successfully
complete the bisimulation proof on the recons-PCFG, and
consequently successfully discharge the
proof obligation
$\{\phi_{{\tt S3:C5}}\}\\({\tt S3\rightarrow S5\rightarrow S3}, {\tt C5\rightarrow C3}) \{l_{S}\indEq{}Clist^{{\tt lnode}}_{m}(l_{C})\}$ in \cref{tab:llproductInv}.
The points-to analysis is described more formally
in \cref{sec:pointsToFormal}.

%Thus, as a part of the invariant inference procedure on the product-CFG
%for the decomposition programs, we also run our points-to analysis on the
%product-CFG to identify points-to invariants at the product-CFG
%nodes.  This improves precision, as shown in the example discussed
%above.
\vspace{-5px}
\subsubsection{Proof discharge algorithm for Category 3 obligations}
\label{sec:cat3summary}

\noindent
Before the start of an equivalence check, a points-to analysis is run on $C$.

During the equivalence check,
to discharge a Category 3 proof obligation $P: {\tt LHS}\Rightarrow{\tt RHS}$ (expressed
in first-order logic), we first replace the recursive
values of program $S$ in the {\tt RHS}
with lifted C values, based on the equalities present in the {\tt LHS}, to
obtain $P_2$.
This is followed by decomposition and RHS-breaking of $P_2$.
If the decomposition
fails, we evaluate $P$ to false (disproven) with a counterexample
that demonstrates the unification failure.

Upon successful decomposition, we
obtain several smaller proof obligations.
To
prove $P$, we require all these smaller proof
obligations to be provable. If any of these smaller proof obligations
is not provable, we are unable to prove $P$.  If we obtain a counterexample
to any of these smaller proof obligations, then that counterexample
also falsifies $P$.
Let $P_3$ represent any such smaller proof obligation. $P_3$, being
a decomposition clause,
must relate atomic expressions on the {\tt RHS}.
If $P_3$ relates two scalar values in the {\tt RHS}, then
it is
a Category 2 proof obligation and can be discharged
using the algorithm in \cref{sec:cat2algo}.

If $P_3$ relates two lifted
expressions (with the same lifting constructor) in
the {\tt RHS},
we check if the reconstruction
programs of the two lifted ADT values being
compared can be proven to be bisimilar (assuming that
{\tt LHS} holds at the correlated entry nodes
of the reconstruction programs).
To improve
the bisimulation
check's precision, we transfer the points-to information at the entry to the
reconstruction programs (obtained from {\tt LHS} which includes
from the points-to results of the C program at the
from-PC where the proof obligation is being discharged)
to the beginning of every iteration
of the unrolling procedure --- this is
achieved by running a points-to analysis
on the reconstruction programs.

These queries
generated by a bisimulation check are discharged
by a recursive call to the proof discharge procedure.
The depth of these recursive calls to the
proof discharge procedure is determined by
the maximum {\em recursion nest depth} (similar
to loop nest depth) of the decomposition
program.

If the bisimilarity check succeeds, the proof procedure returns
true for $P$.
If the bisimilarity check fails,
we imprecisely return false for $P$ (without a counterexample).

Finally, if $P_3$ neither
relates two scalar values, nor relates two
lifted expressions (with the same lifting constructor),
we imprecisely return false for $P$ (without a counterexample).

A more detailed discussion on algorithms introduced in this section
along with their pseudo-code is available in the thesis.